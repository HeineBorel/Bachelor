\section{Spectral Theorem for Unbounded Operators}

Using the results from Section \ref{section4}, we construct a spectral measure
for unbounded normal operators, completing the process started in Section
\ref{section3}. We follow \cite{LesHaupt}.
\begin{thrm}[Spectral Theorem]\label{USpectral}
 Let $T$ be a normal operator on $\hs$.
 Then $\overline{\spe T}^\cb = \spe T \cup \setl \infty \setr$ if, and only if 
 $T $ is unbounded.
 Furthermore $\overline{\spe T}^\cb = \theta(\spe \A)$, where 
 $\A(T) \coloneqq \gen{I, A, B,},~ A \coloneqq (I + T \str T)\inv,
 ~ B \coloneqq TA$. \newline There exists a uniquely determined spectral measure
 \[
  \Phi \colon \Lzero\left( \overline{\spe T }^\cb \right) \to \lh
 \]
 such that
 \begin{enumerate}[(i)]
  \item $
         \setl \infty \setr \text{ is a } \Phi\text{-zeroset},
        $
             
  \item $\Phi_{\id} = T$, where $\id(\infty) \coloneqq 0$, which is arbitrary.
 \end{enumerate}

 In this context, spectral measure shall mean, that the map $\Phi$ restricted to  
 $\Linfty\!\left( \overline{\spe T }^\cb \right) \to \bh$,
 is an isomorphism onto its image. For $f \in 
 \Lzero\left( \overline{\spe T }^\cb \right)$, $\Phi(f)$ has to be 
 a normal, densely defined operator on $\hs$. Furthermore, $\Phi$ must extend the 
 unique Spectral measure $\Phi\colon \cf\left( \overline{\spe T }^\cb \right) \to 
 \bh$ from Proposition \ref{spectraluniqueness}.
\end{thrm}

\begin{proof}
 
 In Proposition \ref{spectraluniqueness}, we defined the inverse Gelfandisomorphism 
 \[
  \Phi \colon \cf ( \theta (\spe \A)) \to \A \subset \lh,~ g \mapsto \G \inv
  (g \circ \theta).
 \]
 It follows that $\Phi$ is a spectral measure for $\cf( \theta (\spe \A)) $ such that
 \begin{align*}
  A = \Phi_a, ~ a\colon \lambda \mapsto \frac{1}{1+ | \lambda| ^2 }, \\
  B = \Phi_b, ~ b\colon \lambda \mapsto \frac{\lambda}{1 + | \lambda | ^2}.
 \end{align*}
 Following \cite[Ch. 4.5]{PedAnaN}, we extend $\Phi$ to measurable functions. 
 As remarked just before Definition \ref{defmeasurable}, this expansion is
 unique. Furthermore, Theorem \ref{maintheorem} holds, proving that
 $\Phi$ is a spectral measure.
 
  To show that $\setl \infty \setr$ is a $\Phi$-nullset,
 it suffices to show that $E_{\setl \infty \setr} =0$. We have
 \[
  A E_{\setl \infty \setr} = \Phi_a \Phi_{1_{\setl \infty \setr}} =
  \Phi_{a 1_{\setl \infty \setr}} = \Phi_0 =0.
 \]
Because $A$ is the inverse of $1 + T\str T$, we get
\[
 E_{\setl \infty \setr} = (1+T\str T)AE_{\setl \infty \setr} = 0.  
\]
Define $\id \colon \theta(\spe \A) \subset \cb \to \C$, via $\id(\infty) =0$. Thus,
\[
\id \in \Lzerophi\text{, and }(1 + | \id| ^2 ) a = 1~~\Phi\text{-a.e.}
\]
Using the fact that $a$ is bounded and Lemma \ref{maintheorem4} $(iv)$, we get
\[
  I = \Phi_1 = \Phi_{(1 +| \id|^2 ) a}= \Phi_{(1 + | \id| ^2 )} \Phi_a = (I + T\str T) A.
\]
Furthermore,
\begin{align*}
  I +TT\str =& \Phi_{(1 +| \id|^2 )} \Phi_a (I+ T T\str)\\
	    =& \Phi_{(1 +| \id|^2 )} A(I + T T\str) \\
     \subset & \Phi_{(1 +| \id|^2 )},
\shortintertext{and using again Lemma \ref{maintheorem5}, we get}
	T= (I +T\str T )A T \subset \Phi_{(1 + | \id |^2 )} TA &=
	\Phi_{(1 + | \id |^2 )} B = \Phi_{(1 + | \id |^2 )} \Phi_b
	= \Phi_{\id}.
\end{align*}
Since $T$ is normal and hence closed, we get that $T = \Phi_{\id}.$
It still remains to show that 
\[
 \theta ( \spe \A ) = \overline{\spe T}^{\cb}.
\]
By Lemma \ref{maintheorem7} , we have 
\[
 \spe T = \spe ( \Phi_{\id} ) = \bigcap_{E_U =I } \overline{\id(U)}^{\C}.
\]
Let $U \subset \theta( \spe  \A )$, such that $U\complement$ 
is a $\Phi$-nullset. Then 
\begin{alignat*}{2}
 \id(U) &= U &&\text{if } \infty \notin U \\
        &= ( U \cup \setl 0 \setr )\setminus \setl \infty \setr &&\text{if }
        \infty \in U,
\shortintertext{giving us}
\overline{\id (U) }^{\C} &= \overline{U}^{\C} &&\text{ if } \infty \notin U \\
			 &= (\overline{U}^{\C} \cup \setl 0 \setr) \backslash \setl \infty \setr
			  &&\text{ if }\infty \in U.
\end{alignat*}
By Corollary \ref{maintheorem3+} , $U\complement$ does not contain any open sets. Thus
\[
 \overline{U}^{\C}  = \theta (\spe \A ) \setminus \setl \infty \setr,
\]
and for each $U$, such that $E_U = I$
\[
 \theta (\spe \A ) \backslash \setl \infty \setr \subset
 \overline{\id ( U )}^{\C} 
 \subset ( \theta ( \spe \A )\cup \setl 0 \setr ) \setminus
 \setl \infty \setr.
\]
For $U_0 \coloneqq \theta ( \spe \A ) \setminus \setl \infty \setr$, it holds 
that $E_{U_0} = I$ and $\id(U_0) = U_0$, and therefore, it follows that
\[
 \spe T = \theta ( \spe \A ) \setminus \setl \infty \setr.
\]
If $T$ is bounded, $\spe T$ is bounded as well, and hence compact, which gives
\[
\overline{\spe T }^{\cb} = \spe T.
\]
Since $T$ is bounded, $I + T\str T = A\inv$ is bounded as well, and since 
$I +T \str T$ is invertible in $\bh$ it is invertible in $\A$ by Proposition
\ref{specinvariant}. Thus $I + T \str T $ lies in $\A$. Assume $\chi_\infty$ is an
element of $\spe(\A)$. $I = (I +T \str T )A$
now implies
\[
 1 = \chi_\infty (I) = \chi_\infty ( I+ T \str T) \chi_\infty ( A) = 
 \chi_\infty ( I + T \str T) \cdot 0 = 0 ,
\]
a contradiction. Thus  $\chi_\infty$ is not an element of $\spe(\A)$, implying
that $ \spe T  = \theta ( \spe \A )$, if $T$  is bounded.
If on the other hand, $\spe T$ is compact in $\C$, then 
\[
 \id \in \Linftyphi \text{, and so } T= \Phi_{\id} \in \bh.
\]
Thus, we have proven 
\[
 T \text{ is bounded } \Leftrightarrow \spe T \text{ compact in } \C.
\]
If now $T$ is unbounded, then $\spe T$ is not compact in $\C$, and hence
\[
\overline{\spe T}^{\cb} = \spe T \cup \setl \infty \setr.
\]
Since $\theta(\spe \A) \subset \cb$ is compact, we get
\[
 \overline{\spe T}^{\cb} = \theta ( \spe \A).
\]
This proves, that there exists a spectral measure 
$\Phi \colon \cf(\overline{\spe T}^{\cb}) \to \lh,$ such that $\Phi_{\id} = T$,
and $\setl \infty \setr$ is a $\Phi$-nullset.
 To check uniqueness, let $\Phi'$ be another spectral measure, such that
$\infty$ is a $\Phi'$-zeroset, and $\Phi'( \id) = T$. To prove $\Phi = \Phi'$,
by the theorem Stone-Weierstraß, we only need to show
\[
 \Phi'_a = \Phi_a = A,~ \Phi'_b = \Phi_b=B.
\]
We know
\[
 T\str T = (\Phi'_{\id}) \str \Phi'_{\id} \subset \Phi_{|\id|^2}.
\]
Since $T\str T$ is normal and hence closed, we have equality.
\begin{align*}
 \Phi'_a ( I + T \str T ) &= \Phi'_a \Phi'_{1 +| \id | ^2} \subset 
 \Phi'_{a (1 + |\id|^2)} = I \\&= \Phi'_{ (1 + |\id|^2)a} 
 = \Phi'_{(1 + |\id|^2)}\Phi'_{a}  = (I - T \str T) \Phi'_a,
 \end{align*}
 that is 
 \begin{align*}
 \Phi'_a &= (I + T \str T)\inv = A \\
 \Phi'_b &= \Phi'_{\id \cdot a} = \Phi'_{\id} A = TA = B.
\end{align*}
\end{proof}


















