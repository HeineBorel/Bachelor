\section{Applications}

Our applications are motivated by quantum physics. Two fundamental operators
in quantum mechanics are the momentum operator $P\coloneqq 
i \frac{\partial}{\partial x}$
, and the position operator $M_x(f)(x) = xf(x)$. 

\begin{rem}
The study of these operators relies heavily on the chosen Hilbert space $\hs$, 
as $\spe M_x$ depends on it.
\end{rem}
\begin{expl}\label{ExplMult}
 

Let $\hs \coloneqq \Ltwo(\left[ 0, 1 \right])$. $M_x \in \bh$ by Hölder's 
inequality: Let $f \in \hs$
\[
 \| M_x(f) \|^2 = \int _0 ^1 x^2 | f(x)|^2 \de x \leq 
 1 \cdot \int _0 ^1 | f(x) |^2 \de x = \| f \| ^2.
\]
Thus $\| M_x\| \leq 1$. Furthermore $M_x$ is self-adjoint, which implies that
$\spe M_x \subset \left[ -1 , 1 \right]$. But for $\lambda > 0$,
$M_{x-\lambda}$ is invertible as $M_{\nicefrac{1}{(x - \lambda)}} \in \bh$, again by
Hölder. Therefore $\spe M_x \subset \left[ 0, 1\right]$. To show 
$\spe M_x \supset \left[ 0, 1\right]$, let $\lambda \in  \left[ 0, 1\right]$.
Again the inverse of $M_{x+ \lambda}$ would be $M_{\nicefrac{1}{(x + \lambda)}}$,
the latter not being bounded, as $\frac{1}{x + \lambda} \notin 
\Ltwo(\left[ 0, 1\right])$. We conclude that $x$ does not a have a preimage
in $\hs$ under $M_{x + \lambda}$. Hence $\spe M_x = \left[ 0 , 1\right]$.

Our Spectral Theorem \ref{boundedfunccalc} states, that the map
$\Phi \colon \cf(\left[0, 1\right]) \to \spe (\gen{M_x, I})$, is an isometry. This 
isometry sends $\id$ to $M_x$, and the constant function $1$ to $I$.
We conclude that $\phi$ is mapped via $\Phi$ to 
$M_{\phi(x)}\colon g \mapsto \phi \cdot g$. In this example, it is obvious that
$\Linfty$ is mapped to $\bh$. If we take $\phi(x) = \nicefrac{1}{x}$, 
$\dom(\phi)$ is by definition, all elements $f \in \hs$, such that
$g \mapsto \gen{g, M_\phi f}$ is continuous in $g$ for all $g \in \hs$.
Since we have such an explicit form of the operator, one readily sees via 
Lemma \ref{maintheorem2} $(ii)$ that
$\dom(\phi) = \setl f \in \hs \bigm | \|M_\phi (f)\|^2 < \infty \setr$.
As seen in Example \ref{ExplFuncCalc}, or by direct calculation, 
$M_\phi$ is the inverse to $M_x$. $M_\phi$ is unbounded, and $M_x$ is 
the bounded inverse to $M_x$. This is an example, where we started with
a bounded operator, and via the Functional Calculus for measurable 
functions, got an unbounded operator

If we change the Hilbert space to $\hs = \Ltwo (\R)$, the spectrum
of our multiplication operator changes to $\spe(M_x) = \R$. Thus, $M_x$ 
is unbounded. Its inverse is again $M_{\nicefrac{1}{x}}$. Note that 
both operators are unbounded, and hence, neither is boundedly invertible.
\end{expl}

\begin{expl}
 
 Let $\hs \coloneqq \Ltwo(\left[ 0, 1 \right])$. 
 The operator we want to consider is the momentum operator 
 $P = i \frac{\partial}{\partial x}$.
 We recall the definition of a normal operator.
 $P$ is called normal if $\dom(P) = \dom(P\str)$ and 
 $\|P\| = \| P \str \|$. To get a normal operator, we
 need to specify boundary conditions on $\dom(P)$. One knows that
 $P\str$ acts the same way as $P$. Therefore, if $P$ is normal,
 it is self-adjoint.
 We compute
 \begin{align*}
  \gen{f, Pg} 
 & =\int _0 ^1 \overline{f(x)} i \frac{\partial}{\partial x}g(x) \dx
  = \overline{f(x)}ig(x)\Bigr|^{1}_{0}
  -\int _0 ^1 i \frac{\partial}{\partial x}\overline{f(x)} g(x) \dx\\
&= i(\overline{f(1)}g(1)- \overline{f(0)}g(0))
+ \int _0 ^1 \frac{\partial}{\partial x}\overline{i f(x)} g(x) \dx\\
&=i(\overline{f(1)}g(1)- \overline{f(0)}g(0)) 
+\gen{Pf, g}.
 \end{align*}
For $P$ to be self-adjoint, the left term from the last line needs 
to vanish. Thus, we get the boundary conditions 
 $\setl f \in H^1(\left[ 0,1 \right]) \bigm| f(0) = f(1) \setr$.
 To determine the spectrum of the operator, we look at 
 the eigenvectors of $(P+\lambda I)$.
 $(P+\lambda I)(f)=f$ reformulates to $i f' = i(\lambda - 1)f$, giving
 $f(x)= \e^{i(\lambda -1 )x}$. This is in the domain of $P$, if 
 $\lambda = 2\pi k$ for $k \in \Z$. We claim that these elements
 already form a Hilbert basis. To see this, we periodically
 extend any
 $f \in \hs \coloneqq \Ltwo(\left[ 0, 1 \right])$ to 
 $\Ltwo(\R)$. Now we can consider functions on the quotient 
 $\R \backslash \Z = \mathbb{S}^1$. For $\Ltwo(\mathbb{S}^1)$,
 $\left(\e^{2\pi i k x }\right)_{k \in \Z}$ forms a Hilbert basis, via
 Fourier expansion. A full proof can be found in 
 \cite[Ch. V.4]{WernerFunkAna}.
 Using the Fourier transform $\mc{F}$, $P$ becomes the multiplication operator
 $M_{2 \pi k}$: For $f \in \dom(P)$, we get
 \begin{align*}
  (\mc{F}(Pf))(k)
  &= \frac{1}{\sqrt{2\pi}} \int _0 ^1 \e^{2\pi i k x}i(\partial_xf)(x)\dx
  = \frac{1}{\sqrt{2\pi}} \int _0 ^1 2 \pi k \e^{2\pi i k x}f(x)\dx \\
  &= 2 \pi k (\mc{F}f)(k).
 \end{align*}
The Fourier transform maps $\Ltwo(\mathbb{S}^1)$ isometrically
to $\ell^2(\Z)$, found in \cite[p. 205]{TaylorPDEI}. $\spe(M_{2\pi k}) = 2\pi \Z$, as 
$(M_{2 \pi k } + \lambda I)\inv = (M_{2 \pi k + \lambda})\inv
= M_{(2 \pi k + \lambda)\inv}$, for $\lambda \notin 2\pi \Z$.

The map $\Phi \colon \Lzero(\spe(P)) \to \lh, ~
f \mapsto \mc{F}M_{f(2\pi k)} \mc{F}\inv$ satisfies the conditions
of being a spectal measure for the operator $P$. By Theorem \ref{USpectral}
it is the unique spectral measure for $P$. 
We can therefore apply the arguments from Example
\ref{ExplMult}, to $M_{2 \pi k}$, and transform back.

\end{expl}

The fact that we could find another Hilbert space, such that our operator
acts as a multiplication operator, was not mere chance. A more general 
version of the spectral theorem states the following:

\begin{thrm}[Spectral Theorem, multiplication operator]
\label{uspecmult}
Let $T$ be a normal operator on a Hilbert space $\hs$. There exists
a measure space $(\Omega, \Sigma, \mu)$, a measurable function 
$f\colon \Omega \to \C$, and a unitary operator $U\colon \hs \to \Ltwo(\mu)$,
such that
\[
UTU\str \phi = f \cdot \phi = M_f(\phi) \text{ for }\phi \in \dom(M_f) 
 = \setl \phi \in \Ltwo(\mu) \bigm | f\phi \in \Ltwo(\mu) \setr.
\]
% \begin{enumerate}[\normalfont (a)]
%  \item $x \in \dom(T)$ if, and only if $f\! \cdot\! U(x) \in \Ltwo(\mu)$.
%  \item $UTU\str \phi = f \cdot \phi = M_f(\phi)$ for 
%  $\phi \in \dom(M_f) 
%  = \setl \phi \in \Ltwo(\mu) \bigm | f\phi \in \Ltwo(\mu) \setr$.
% \end{enumerate}
\end{thrm}
The proof for bounded self-adjoint $T$ can be found in \cite[Ch. VII.1]{WernerFunkAna}.
Here, we sketch how one can extend this proof for normal operatorrs.

\begin{proof}
 

First assume, that $x$ is a $\str$-cyclic vector for $T$, meaning that 
the span of $\setl {T\str} ^k T^n x\bigm | k,n \in \N \setr$ lies dense in $\hs$. 
% By 
% the Riesz--Markov--Kakutani representation theorem
% \cite[Theorem 6.3.4]{PedAnaN}, $\m{x,x}$ 
Consider the map $V\colon\cf(\spe(T)) \to \hs$, 
$\phi \mapsto \Phi_\phi x$, where $\Phi$ is the unique spectral 
measure obtained in Theorem \ref{USpectral}. Then
\begin{gather*}
 \int |\phi|^2 \dm{x,x} = \int \overline{\phi}\phi \dm{x,x}
 = \gen{\Phi_{\overline{\phi} \phi} x, x}= \gen{\Phi_\phi x, \Phi_\phi x} 
 = \|\Phi_\phi x \|^2,
\end{gather*}
implying that $V$ can be isometrically extended to a map 
$\overline{V}\colon \Ltwo(\m{x,x}) \to \hs.$ As $x$ is a cyclic vector, 
the image of $\overline{V}$ is the whole Hilbert space $\hs$. Thus, 
$\overline{V}$ is a surjective isometry, and therefore unitary. For
$\phi \in \Ltwo(\m{x,x})$, such that $\id \cdot \phi$ is still 
square integrable, we calculate
\[
 T(\overline{V}(\phi))= T(\Phi_\phi x) = (T\circ \Phi_\phi)x = (\Phi_{\id} \Phi_\phi) x
 = \overline{V}(\id \cdot \phi),
\]
where the last equality holds by Lemma \ref{maintheorem4}. This gives
\[
 (\overline{V}\inv T \overline{V})\phi = \id \cdot \phi.
\]
Now $U\coloneqq  \overline{V}\inv = \overline{V}\str$, and $\id$ satisfy 
the conditions of the theorem.

Unfortunately, we can not expect $T$ to have a cyclic vector. However, we can 
decompose $\hs$ into cyclic subspaces via an argument using Zorn's Lemma.
For this argument to work, we need that $\bigcap_k \dom(T^k)$ is dense in $\hs$,
which is an easy application of Theorem \ref{USpectral}.
We adopt the notation from \cite[pp. 337]{WernerFunkAna}, writing
$\hs = \bigoplus_2 \hs_i$, $x = (x_i)$ for a sum of pairwise orthogonal subspaces, such that
the closure of the span is the whole space. Once we have a decomposition
$\hs = \bigoplus_2 \hs_i$ into cyclic subspaces, we apply the 
considerations above, and get
unitary maps $U_i \colon \Ltwo(\m{x_i, x_i}) \to \hs_i$, 
($U_i T_i U_i \str)\phi_i = f_i \cdot \phi_i$. We now take the direct sum
(defined in \cite[214L]{FremMeasureTheo}) of the measure spaces $(\spe(T_i), \m{x_i,x_i})$,
and obtain a new measure space ($\dot\bigcup \spe(T_i),\m{}$).
We write $f = (f_i)$, $f(x) = f_i(x)$ if $x \in \spe(T_i)$. 
Define a new operator $U\colon\hs \to \Ltwo(\m{})$, via 
$U((x_i)) = (U_i(x_i))$. It follows that $U$ is unitary, and 
$(UTU\str)(\phi) = f\phi$, finishing the proof.
\end{proof}

The theorem tells us, that the multiplication operator is indeed 
the most general, and in a certain sense, the only example of a normal
operator.























