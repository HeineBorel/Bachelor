\section{Examples}

Our examples are motivated by quantum physics. Two fundamental operators
in quantum mechanics are the momentum operator $P\coloneqq 
i \frac{\partial}{\partial x}$
, and the position operator $M_x(f)(x) = xf(x)$. 

\begin{rem}
The study of these operators relies heavily on the chosen Hilbert space $\hs$, 
as $\spe M_x$ depends on it.
\end{rem}
\begin{expl}
 

Let $\hs \coloneqq \Ltwo(\left[ 0, 1 \right])$. $M_x \in \bh$ by Hölder's 
inequality: Let $f \in \hs$
\[
 \| M_x(f) \|^2 = \int _0 ^1 x^2 | f(x)|^2 \de x \leq 
 1 \cdot \int _0 ^1 | f(x) |^2 \de x = \| f \| ^2.
\]
Thus $\| M_x\| \leq 1$. Furthermore $M_x$ is selfadjoint, which implies that
$\spe M_x \subset \left[ -1 , 1 \right]$. But for $\lambda > 0$,
$M_{x-\lambda}$ is invertible as $M_{\nicefrac{1}{(x - \lambda)}} \in \bh$, again by
Hölder. Therefore $\spe M_x \subset \left[ 0, 1\right]$. To show 
$\spe M_x \supset \left[ 0, 1\right]$, let $\lambda \in  \left[ 0, 1\right]$.
Again the inverse of $M_{x+ \lambda}$ would be $M_{\nicefrac{1}{(x + \lambda)}}$,
the latter not being bounded, as $\frac{1}{x + \lambda} \notin 
\Ltwo(\left[ 0, 1\right])$. We conclude that $x$ does not a have a preimage
in $\hs$ under $M_{x + \lambda}$. Hence $\spe M_x = \left[ 0 , 1\right]$.

Our spectral theorem \ref{boundedfunccalc} now states, that we have
an isometry $\Phi : \cf(\left[0, 1\right]) \to \spe (\gen{M_x, I})$. This 
isometry sends $\id$ to $M_x$, and the constant function $1$ to $I$.
We conclude that $\phi$ is mapped via $\Phi$ to 
$M_{\phi(x)}: g \mapsto \phi \cdot g$. In this example, it is obvious that,
$\Linfty$ is mapped to $\bh$. If we take $\phi(x) = \nicefrac{1}{x}$, 
$\dom(\phi)$ is by definition, all elements $f \in \hs$, such that
$g \mapsto \gen{g, M_\phi f}$ is continuous in $g$ for all $g \in \hs$.
Since we have such an explicit form of the operator, one readily sees via 
Lemma \ref{maintheorem2} $(ii)$ that
$\dom(\phi) = \setl f \in \hs \bigm | \|M_\phi (f)\|^2 < \infty \setr$.
As seen in Example \ref{ExplFuncCalc}, or by direct calculation, 
$M_\phi$ is the inverse to $M_x$. $M_\phi$ is unbounded, and $M_x$ is 
the bounded inverse to $M_x$. This is an example, where we started with
a bounded operator, and via the Functional Calculus for measurable 
functions, got an unbounded operator

If we change the Hilbert space to $\hs = \Ltwo (\R)$, the spectrum
of our multiplication operator changes to $\spe(M_x) = \R$. Thus, $M_x$ 
is unbounded. Its inverse is again $M_{\nicefrac{1}{x}}$. Note that 
both operators are unbounded, and hence, neither is boundedly invertible.
\end{expl}

\begin{expl}
 
 Let $\hs \coloneqq \Ltwo(\left[ 0, 1 \right])$. 
 The operator we want to consider is the momentum operator 
 $P = i \frac{\partial}{\partial x}$.
 We recall the definition of a normal operator.
 $P$ is called normal if $\dom(P) = \dom(P\str)$ and 
 $\|P\| = \| P \str \|$. To get a normal operator, we
 need to specify boundary conditions on $\dom(P)$. One knows that
 $P\str$ acts the same way as $P$. Therefore, if $P$ is normal,
 it is selfadjoint.
 We compute
 \begin{align*}
  \gen{f, Pg} 
 & =\int _0 ^1 \overline{f(x)} i \frac{\partial}{\partial x}g(x) \dx
  = \overline{f(x)}ig(x)\Bigr|^{1}_{0}
  -\int _0 ^1 i \frac{\partial}{\partial x}\overline{f(x)} g(x) \dx\\
&= i(\overline{f(1)}g(1)- \overline{f(0)}g(0))
+ \int _0 ^1 \frac{\partial}{\partial x}\overline{i f(x)} g(x) \dx\\
&=i(\overline{f(1)}g(1)- \overline{f(0)}g(0)) 
+\gen{Pf, g}.
 \end{align*}
For $P$ to be selfadjoint, the left term from the last line needs 
to vanish. Thus, we get the boundary conditions 
 $\setl f \in W^{1,2}(\left[ 0,1 \right]) \bigm| f(0) = f(1) \setr$.
 To determine the spectrum of the operator, we look at 
 the eigenvectors of $(P+\lambda I)$.
 $(P+\lambda I)(f)=f$ reformulates to $i f' = i(\lambda - 1)f$, giving
 $f(x)= \e^{i(\lambda -1 )x}$. Since $\e^{i(\lambda -1 )x}$ is not in the
 domain of $P$, we see that $P$ is injective. Thus, if an element has
 a preimage, it is unique. One has $(P+\lambda I)(x) = i + \lambda x$.
 Since $x \notin \dom(P)$, we see that $i + \lambda \id$ has no preimage.
 Because $\lambda$ was arbitrary, $\spe(P) = \R$.
 
 The algebra $\A(P) = \gen{I , A, B, B\str}$, $A = (I + P^2)\inv$,
 $B = TA$. $P^2 = -\frac{\partial ^2}{\partial x^2} = \Delta$ the Laplacian.
 This is a positive, elliptic differential operator. 
 
 
 To avoid boundary value problems, we consider $\hs = \Ltwo(\R)$. Then
 $P$, with domain $H^2(R)$ is a selfadjoint operator. Using the Fourier 
 transform, $P$ is transformed into a multiplication operator. This
\end{expl}






























