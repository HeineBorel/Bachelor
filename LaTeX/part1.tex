
%\documentclass[a4paper,10pt]{scrartcl}


 

\section{Preliminaries}
In this section, I want to state a few theorems and definitions, which will be
used later on. If no proof is given, a reference will be stated nevertheless. 
For an unital algebra $\A$, we denote the invertible elements by $\GL(\A)$.
\begin{defi}[Spectrum]
Let $\A$ be a unital commutative Banach algebra. For $A$ an element of $\A$,
the \textit{spectrum} of $A$ in $\A$, is defined as 
\begin{gather*}
\spe_\A(A)\coloneqq \setl z \in \C \mid A-z  I \notin \GL(\A) \setr.
\shortintertext{The \textit{spectrum} of $\A$ is defined as}
\spe(\A) \coloneqq \setl \chi :\A \to \C \mid \chi \in 
\Hom_{\cAlg{\C}}(\A, \C), \chi \neq 0 \setr.
\end{gather*}
\end{defi}
 
% and by $\A''$ the double dual $\Hom_\C(\Hom_\C(\A, \C))$. 
\begin{prop}\label{SpecSurj}
Using the notation above, 
\[
\spe_\A(A) = \setl \chi(A) \bigm | \chi \in \spe \A \setr .
\]
In other words, the spectrum of an element is the image of that element
under the spectrum of the algebra.
\end{prop}
The proof can be found in \cite[Ch. 4.2]{PedAnaN}.
By $\A'$ we denote the dual space $\Hom_\C(\A, \C)$, endowed with the 
weak-$*$ topology.


\begin{thrm}[Gelfand--Naimark] 

Let $\A$ be a commutative $C\str$-algebra. If $\spe(\A)$ is equipped 
with the supspace
topology of $\A'$, it becomes a compact space, with a canonical isometric, involutive,
surjective algebra homomorphism
\[
  \G : \A \to \cf(\spe(\A)), ~ A \mapsto ( \hat{A} \coloneqq \G(A):
  \spe(\A) \to \C , ~ \gamma \mapsto \gamma(A)).
\]
\end{thrm}
The proof can be found in \cite[Ch. 4.3]{PedAnaN}. The map in the theorem is
the so called Gelfand transform.% named after Israel Moissejewitsch Gelfand (1941).

The reader not familiar with closed operators is advised to revisit the
basic definitions found in \cite[Ch. 10]{ConFuncAna}.  When we speak of a 
linear operator, we always mean a linear operator.
Let $T$ be a operator on some Hilbert space $\hs$.
We will adopt the notation from \cite[Ch. 10]{ConFuncAna}, that is, $T$ is not even
assumed to be defined for any non zero element. Later on, we will only 
concern ourselfs with densly defined, closed operators. By $\domt$ and
$\rant$ we denote the domain and the range
of the operator, respectively. The set of bounded linear operators will be
called $\bh$, in
contrast to $\lh$, which is the set of all linear operators.
The set of closed operators is denoted by $\ch$.

\begin{defi}
  If $T$ and $S$ are operators on $\hs$, we say $T$ \textit{extends} $S$,
  if $\dom(S)\subset
  \domt$ and $ Tx = Sx$ for all $x$ in $ \dom (S)$. We write $S \subset T$.
\end{defi}

Let $S$ and $ T$ be operators on $\hs$. By $S + T$,  we denote
the operator with domain 
$\dom(S) \cap \domt$ and rule $(S + T)x = Sx + Tx$. Note that this does
not give $\lh$ the structure of a vector space, for 
$(S + T) - T \neq S$ since
$\dom(T)\cap \dom(S) \neq \dom(S)$.
$TS$ is defined to be the operator with domain $S\inv \domt$. The reader should
be aware, that with the just defined operations, $\lh$ does not admit the
structure of an algebra, and as previously remarked, not even that of a
vector space. This is one
of the reasons, why one has to be careful while working with unbounded
operators. 

Since closed operators are not defined on all of $\hs$,
there is no obvious definition of an inverse to such an operator.
However, we have the following

\begin{defi}
 Let $T$ denote a closed operator on $\hs$. We say that $T$ is 
 \textit{boundedly invertible} if
 $T:\dom(T) \to \ran(T)= \hs$ is a bijection, and $T\inv :\hs \to \dom(T)$
 is continuous. $T\inv$ is called the \textit{bounded inverse} of $T$. As $T\inv$ is
 continuous, the closed graph theorem implies the boundedness of $T\inv$.
\end{defi}

\begin{rem}
 If $T$ is boundedly invertible, then the inverse is unique.
\end{rem}

% 
% \begin{lem}
%  If $T$ is boundedly invertible, the bounded inverse $S= T\inv$ is unique. 
% \end{lem}
% 
% \begin{proof}
% 
%  Let $S, S'$ be bounded inverses of $T \in \ch$. Then
%  \[
%   0 = TS - TS' = T(S - S').
%  \]
% Hence, $ \ran(S - S') \subset \ker(T)$. But $\ker(T) = 0$, since $ST \subset \id$, which implies that $S = S'$.
% \end{proof}


\begin{lem}
 If $T \in \ch$, $S \in \bh$ and $ TS = \id$ then $S$ is the bounded inverse of $T$.
\end{lem}

\begin{proof}
 It remains to show that $T$ is a bijection.
 Surjectivity is obvious. Note that $\ker(S) =0$, which implies that 
 $\ker(T)=0$ as well.
 \end{proof}

For $A, B, \dots \in \A$, we denote by $\gen{A, B, \dots}$ the
$C\str$-subalgebra of $\A$, generated by the elements $A, B, \dots$


% \newpage nxfnsbxfvzmuphdw
% 
% \includepdf{Background.pdf}

 
















