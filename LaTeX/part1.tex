
%\documentclass[a4paper,10pt]{scrartcl}


 

\section{Preliminaries}
Throughout this paper, I will use some theorems, which will be explained in this section.
$\A$ will always denote a complex, unitary Banach algebra, with the unit denoted by $I$. We have the 
following
\begin{defi}[Spectrum]
For $A \in \A$ the \textit{spectrum} of $A$ in $\A$, is defined as 
$\spe(A)\coloneqq \setl z \in \C \mid A-z  I \notin \GL(\A) \setr$.
The \textit{spectrum} of $\A$ is defined as $\spe(\A) \coloneqq \setl \chi :\A \to \C \mid \chi \in
\Hom_{\cAlg{\C}}(\A, \C) \setr  $.
\end{defi}

 
With $\A'$ we denote the dual space $\Hom_\C(\A, \C)$, and by $\A''$ the double dual $\Hom_\C(\Hom_\C(\A, \C))$. 

\marginparr{umschreiben}
\begin{thrm}[Gelfand--Naimark] 

Let $\A$ be commutative staralgebra. Then $\spe(\A)$ with the supspace topology of $\A'$ is a compact space, with a canonical isometric, involutive, surjective algebra homomorphism
\[
  \G : \A \to \cf(\spe(\A)), ~ A \mapsto ( \hat{A} \coloneqq \G(A): \spe(\A) \to \C , ~ \gamma \mapsto \gamma(A)).
\]
\end{thrm}
The map in the theorem is the so called Gelfandtransform, named after Israel Moissejewitsch Gelfand (1941).

\marginparr{ueberpruefen ob das stimmt}
$T$ will always denote a closed operator between some Hilbert spaces $\hs$ and $ \tilde{\hs}$. Most of the time, we will be in the situation of $\hs = \tilde{\hs}$. The space of all such operators will be called $\ch$. By $\domt$ and $\rant$ we denote the domain respectively range of the operator. The bounded linear operators will be called $\bh$ in contrast to $\lh$, which are linear operators which need not be bounded and therefore continous.
\marginparr{Mehr ueber unbeschraenkte Operatoren?}

\begin{defi}
  For $T, S \in \lh$ we say $T$ \textit{extends} $S$, if $\dom(S)\subset \domt$ and $ Tx = Sx$ for $x \in \dom (S)$. We write $S \subset T$.
\end{defi}

For $S, T \in \ch$, let $S + T$ be the operator with domain $\dom(S) \cap \domt$ and Vorschrift $(S + T)x = Sx + Tx$. Note that this does not give $\ch$ the structure of a vector space, since $S +T \notin \ch$ and  $(S + T) - T \neq S$ because $\dom(S) \cap \domt$ need not be densely defined. $TS$ is the linear operator with domain $S\inv \domt$, and the obvious Vorschrift. The reader shoulb be aware, that with the just defined operations, $\ch$ does not admit the structure of an algebra, not even that, of a vector space. This is one of the reasons, why one has to be careful while working with unbounded operators. Since closed operators are not defined on all of $\hs$, there is no intrinsic defnition of an inverse to such an operator, since we do not want map from $I: \domt \to \domt$. Hence, we have the following

\begin{defi}
 Let $T \in \ch$. We say that $T$ is \textit{boundedly invertible} if there exists a bounded linear operator $S: \hs \to \hs$ such that $TS = \id$, and $ST \subset \id$. We call $S$ the \textit{bounded inverse} of $T$.
\end{defi}

\begin{lem}
 If $T$ is boundedly invertible, the bounded inverse $S= T\inv$ is unique. 
\end{lem}

\begin{proof}

 Let $S, S'$ be bounded inverses of $T \in \ch$. Then
 \[
  0 = TS - TS' = T(S - S').
 \]
Hence, $ \ran(S - S') \subset \ker(T)$. But $\ker(T) = 0$, since $ST \subset \id$, which implies that $S = S'$.
\end{proof}


\begin{lem}
 If $T \in \ch$, $S \in \bh$ and $ TS = \id$ then $S$ is the bounded inverse of $T$.
\end{lem}

\begin{proof}
 \begin{align*}
 &\quad & TS &= \id &\\
 &\Leftrightarrow & STS &= S &\\
 &\Leftrightarrow & (ST - \id)S &= 0 &\\
 &\Leftrightarrow & \ran(S) &\subset \ker(ST - \id). & 
\end{align*}
 But $\ran(S)$ is dense in $\hs$, and therefore $ST = \id|_{\domt}$, which implies $ST \subset \id$.
\end{proof}


\marginparr{star-subalgebra, oder sub-staralgebra, oder starsubalgebra, oder *-subagebra...?}



% \newpage nxfnsbxfvzmuphdw
% 
% \includepdf{Background.pdf}

 
















