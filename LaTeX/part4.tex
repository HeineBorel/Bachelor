
\section{Spectral measures}

\begin{defi}[Spectral measure]
 
 Let $X$ be a compact space and let
 \[
  \Phi : \cf (X) \to \lh
 \]
be a map. $\Phi$ is called a \textit{spectral measure}, if its image 
$\A \coloneqq \Phi(\cf (X))$ is a commutative star-subalgebra of $\lh$ and 
$\Phi$ induces an isomorphism onto its image.
\end{defi}

\begin{rem}
 By isomorphism we mean that
 \begin{enumerate}
  \item $\Phi$ is an involutive algebrahomomorphism,
  \item $\Phi$ is a bijection onto A,
  \item $\Phi$ is an isometry: $\| \Phi f \| = \| f \|_\infty$.
 \end{enumerate}

\end{rem}

% Hier noch das Beispiel uebernehmen ?

Let $\m$ be a positive Radon measure on our compact space $X$, meaning a
continuous linear form on $\cf (X)$, such that $\m(f) = 0$ implies that $f=0$. Set $\hs \coloneqq L^2(\mathfrak{m})$.
We define 
\begin{align*}
 \Phi : \cf(X) \to \mathscr{L}({\Ltwom}) \\
 f \mapsto (g \mapsto f\cdot g).
\end{align*}

For all $g , h \in \hs$ define
\[
 \m_{g, h} (f) \coloneqq \gen{g, \Phi_f h}.
\]
We then have that the map 
\[
 f \mapsto \m_{g,h} (f)
\]
is a linear form on $\cf (X)$ for every pair $(g, h) \in \hs \times \hs$.

\begin{thrm}
For all $g, h,k \in \hs$, $\alpha, \beta \in \C$ and $f, \phi, \psi \in 
\cf(X)$, it holds that $\m_{g,h}$ is a Radon measure on $X$ with the following 
properties:
\begin{enumerate}[{(}i{)}]
 \item $\| \m_{g,h}\| \leq \|g\| \|h\|$
 \item $\m_{\alpha g + h, \beta k} = \overline{\alpha}\beta \m_{g,k} + 
 \beta \m_{h,k}$
 \item $\overline{\m}_{g,h} = \m_{h,g}$, $\overline{\m}_{g,h} : 
 f \mapsto \overline{\m}_{g,h}(f) = \overline{\m_{g,h}(\overline{f})}$
 \item $\m_{g,g} \geq 0$
 \item $\m_{\Phi_\phi g, \Phi_\psi h} = \overline{\phi} \psi \m_{g,h}.$
\end{enumerate}

\end{thrm}

\begin{proof}
 \begin{enumerate}[(i)]
  \item \begin{align*}
         \| \m_{g,h} \| &= \sup\limits_{\| f \| _\infty \leq 1} | \m_{g,h}(f) | \\
			&= \sup\limits_{\| f \| _\infty \leq 1} | \gen{g, \Phi_f h} | \\
			&\leq \sup\limits_{\| f \| _\infty \leq 1} \| g \| \|\Phi_f \| \| h\| \\
			&= \|g\| \|h\|
        \end{align*}
  since $\|\Phi_f \| =\| f \|_\infty$.
 \item follows immediately from the linearity of $\gen{\cdot , \cdot}$
 
 \item $\overline{\m}_{g,h}(f) = \overline{\gen{g, \overline{f}h}} = 
       \gen{\overline{f}h ,g} =	f\gen{h,g} = \gen{h, fg} = \m_{h,g}(f).$
	
 \item Let $\phi \geq 0$. Because $\Phi$ is algebrahomomorphism we have 
	\[
	  \Phi_\phi = \Phi_{\sqrt{\phi}} \Phi_{\sqrt{\phi}} = 
	  \Phi_{\overline{\sqrt{\phi}}} \Phi_{\sqrt{\phi}} =
	  \Phi_{\sqrt{\phi}}\str \Phi_{\sqrt{\phi}}.
	\]
	Therefore $\m_{g,g}(\phi) = \gen{g, \Phi_\phi g} = \gen{\Phi_{\sqrt{\phi}}g,
	\Phi_{\sqrt{\phi}}g} \geq 0. $
	
 \item	$\m_{\Phi_\phi g, \Phi_\psi h} (f ) = \gen{\Phi_\phi g, \Phi_f 
	\Phi_\psi h} = \gen{g , \Phi_{\overline{\phi} \psi f} h } = \overline{\phi}
	\psi \m_{g,h}(f)$ 

\end{enumerate}

\end{proof}

Let $\A ^c $ denote the commutant of $\A$.

\begin{lem}
 Let $S \in \lh$. $S \in \A ^c$ if and only if $\m_{g, Sh} = \m_{S \str g, h}$ 
 for all $g, h \in \hs$. 
\end{lem}

\begin{proof}
 Follows from definition.
\end{proof}

\begin{defi}
 $N \subset X$ is called a $\Phi$-set of measure zero, or $\Phi$-zeroset, if
 $N$ is a zeroset of $| \m_{g,h}|$ for all $g,h \in \hs$, that is 
 \[
  | \m_{g, h} | (N) =0.
 \]
 Note that $| \m_{g, h} |$ is a real valued Radon measure. Hence we can talk 
 about measurability of sets in the sense of Pedersen.
 \marginparr{Mehr erklaeren zu Messbarkeit?}
 \marginparr{Hier noch Bibliographirefereny yu pedersen einfuegen}
 
 A function $f: X \to \C$ is called $\Phi$-measurable if $f$ is 
 $|\m_{g,h}|$-measurable. Denote the set of all measurable functions 
 $\Lzerophi)$. 
 \begin{align*}
   \Linftyphi \coloneqq \setl f \in \Lzerophi \bigm |  \|f \| _\infty < \infty \setr
  \intertext{where, }
  \| f \| _\infty \coloneqq  \inf \setl \lambda > 0 \bigm |
    |f| \leq \lambda,  \phiae\setr.
 \end{align*}
 \[
 \mblsx \coloneqq \setl A \subset X \bigm | 1_A \in \Lzerophi \setr
 \]

 We would like to expand $\Phi$ from just continuous functions.
 Let $f \in \Lzerophi$,
 \begin{align*}
 D(f) &\coloneqq \big\lbrace h \in \hs \bigm | f \in \Lone  (\m_{g,h}) \text{ for all } g
 \in \hs\\
 &~~~~~~~~\text{ and } g \mapsto \int f \dm_{g, h} \text{ is continuous} 
 \big\rbrace\\
 &=  \big\lbrace h \in \hs \bigm | f \in \Lone (\m_{g,h}) \text{ for all } g \in
 \hs \\
 &~~~~~~~~\text{ and } \exists k \in \hs \text{ such that } \int f \dm_{g,h}
 = \gen{g, k} \big\rbrace
 \end{align*}
where the second inequality is due to the Riesz representation theorem.
Using this, we define 

\[
\Phi_f h \coloneqq k(h, f) = k \text{ for } h \in D(f).
\]

In other words we have 

\[
\int f \dm_{g,h} = \gen{g, \Phi_f h} = \gen{g, k}.
\]

By linearity of $\gen{\cdot, \cdot}$, $\Phi_f$ is linear as well.
\end{defi}
\marginparr{Dichtheit der richtige Begriff hier?}

\begin{rem}

If we want to proof claims about $\Lzerophi$ it is enough to show them for 
$\cf(X)$, by denseness of the latter space.
 
\end{rem}

\begin{lem}
 
  For all $f \in \Lzerophi$ and $h \in D(f)$, $g\in \hs$ it holds that
 \[
 \m_{g, \Phi_f h} = f\m_{g,h}
 \]

\end{lem}

\begin{proof}
 By the remark, let $\phi \in \cf(X)$.
 \begin{align*}
   \m_{g, \Phi_f h}(\phi) &= \gen{g, \Phi_\phi (\Phi_f h)} 
			  = \gen{\Phi_{\overline{\phi}} , \Phi_f h}
			  = \int f \dm_{\Phi_{\overline{\phi}} g, h}\\
			  &= \int f \phi \dm_{g,h}
			  = \left( f \m_{g,h} \right)(\phi)
 \end{align*}

\end{proof}


The proof of the theorem will be split into several smaller parts, each including their own lemmata.


\begin{lem}
 
 If $f \in \Linftyphi$, then $D(f) = \hs$, $ \Phi_f \in \lh$ and
 \[
 \| \Phi_f \| = \| f\|_\infty = \inf \setl \alpha > 0 \bigm | \alpha \geq |f|, \phiae\setr.
 \]

\end{lem}
\begin{proof}
   
   
   
   Let $f \in \Linftyphi$.
   We have to show that $D(f) = \hs$.
   Let $h \in \hs$. For $g \in \hs$ one gets:
   \begin{align*}
     \left | \int f \dm_{g, h} \right | &\leq \int |f| \text{d}  |\m_{g,h}| \\
			   &\leq \| f\| _\infty \int \text{d}  |\m_{g,h}|\\
			   &= \| f\|_\infty \| \m_{g,h} \|  \\
			   &\leq \| f\|_\infty \|h\| \|g\| 
\shortintertext{which implies}
g &\mapsto \int f \dm_{g,h}
\end{align*}
is continuous and $f$ an element of $ \mscr{L}^1(\m_{g,h})$.
We used that 


\[
\|\m_{g,h}\|=\sup_{\| \phi \|_\infty \leq 1} |\m_{g,h}| = \sup_{ \substack{| \phi | \leq 1\\ \phi \in \cf(X)}} |\m_{g,h}|
 =  \int \text{d}  |\m_{g,h}| = |\m_{g,h}|(1) .
\]

This shows that the assignment $g \mapsto \int f \dm_{g,h}$ defines a continuous map.
\marginparr{g,h aendern?}
\begin{align*}
  \| \Phi_f \| &= \sup_{\| g\|, \| h\| \leq 1}  \left| \gen{g, \Phi_f h} \right | \\
  &= \sup_{\| g\|, \| h\| \leq 1}  \left| \int f \dm_{g,h} \right | \\
  &\leq \|f\|_\infty
\end{align*}

Other inequality, will be proved later.
\marginparr{Den Beweis hier einfuegen}

\begin{rem}
 If$ f_n \to f \text{ in } \Linftyphi, $ then
\[
\Phi_f = \lim \Phi_{f_n} \text{ in } \lh.
\]
We can therefore define $f(T)$ for all bounded $f$.
\end{rem}
\marginparr{Unklar warum man das definieren kann}

\end{proof}

\begin{lem}
 
Let $f \in \Lzerophi, (f_n)$ a net in $\Linftyphi$, and $\alpha, \beta \geq 0$ such that
 \begin{align*}
   | f_n | \leq \alpha |f| + \beta \text{ for all } n \\
   f_n \to f \phiae.
 \end{align*}
Then the following statements about $h \in \hs$ are equivalent:
\begin{enumerate}[(i)]
 \item $h \in D(f)$
 \item $\int \str | f| ^2 \dm_{h,h} < \infty$
 \item $(\Phi_{f_n}h)$ converges in $ \hs$
\end{enumerate}
One then has
\begin{align*}
  \| \Phi_f h \| ^2 &= \int |f|^2 \dm_{h, h } \text{ and }\\
  \Phi_{f }h &= \lim \Phi_{f_n} h.
\end{align*}
For example, one could take the net $f_n = 1_{A_n}$, where $A_n = \setl x \in X \bigm | | f(x) | \leq n \setr$.
\end{lem}

\begin{proof}
 $(i) \Rightarrow (ii)$
 
 Let $h \in D(f)$.
 \begin{align*}
   \infty > \|\Phi_f h\| ^2 &= \gen{ \Phi_f , \Phi_f h} \\
   &= \int f \dm_{\Phi_f h, h} \\
   &= \int f \overline{f} \dm_{h,h} \\
   &= \int | f | ^2 \dm_{h,h}.
 \end{align*}

 
  $(ii) \Rightarrow (iii)$
  Let $f \in \Ltwophi$. Then 
  \begin{align*}
    \gen{g, (\Phi_{f_m} - \Phi_{f_n})h} &= \int f_m \dm_{g,h} - \int f_n \dm{g,h} \\
    &= \int (f_m - f_n) \dm{g,h} \\
    &= \gen{g, \Phi_{(f_m - f_n)} h }
  \end{align*}
  
  \begin{align*}
    \|(\Phi_{f_m} - \Phi_{f_n}) h \| &=   \sup_{\|g\| \leq 1} 
    |(\gen{g, (\Phi_{f_m} - \Phi_{f_n})h}|^2 \\
    &= \sup_{\|g\| \leq 1} |\gen{g, \Phi_{(f_m - f_n)}h}| ^2 \\
    &= \| \Phi_{(f_m - f_n)} \| ^2\\
    &= \int |f_m - f_n|^2 \dm_{h,h}
  \end{align*}


\end{proof}

























\begin{thrm}[Main theorem]
  \leavevmode
  \begin{enumerate}
   \item   

 If $f \in \Linftyphi$, then $D(f) = \hs$, $ \Phi_f \in \lh$ and
 \[
 \| \Phi_f \| = \| f\|_\infty = \inf \setl \alpha > 0 \bigm | \alpha \geq |f|, \phiae\setr.
 \]

 
\item 
Let $f \in \Lzerophi, (f_n)$ a net in $\Linftyphi$, and $\alpha, \beta \geq 0$ such that
 \begin{align*}
   | f_n | \leq \alpha |f| + \beta \text{ for all } n \\
   f_n \to f \phiae.
 \end{align*}
Then the following statements about $h \in \hs$ are equivalent:
\begin{enumerate}[(i)]
 \item $h \in D(f)$
 \item $\int \str | f| ^2 \dm_{h,h} < \infty$
 \item $(\Phi_{f_n}h)$ converges in $ \hs$
\end{enumerate}
One then has
\begin{align*}
  \| \Phi_f h \| ^2 &= \int |f|^2 \dm_{h, h } \text{ and }\\
  \Phi_{f }h &= \lim \Phi_{f_n} h.
\end{align*}
For example, one could take the net $f_n = 1_{A_n}$, where $A_n = \setl x \in X \bigm | | f(x) | \leq n \setr$.

\item
For each $f \in \Lzerophi$, $\Phi_f$ is a normal Operator, and we have
\[
 \Phi_f \str = \Phi_f. 
\]
If $f$ is real valued, then $\Phi_f$ is self-adjoint, and $\Phi_{1_A} \eqqcolon E_A$ for $ A \in \mblsx$ is an orthogonal projection. 

\item 
For each $f, g \in \Lzerophi$, $\alpha \in \C$, we have

\begin{enumerate}[(i)]
  \item $\Phi_{\alpha f} = \alpha \Phif$
  \item $ \overline{\Phif + \Phi_g} = \Phi_{f + g}$
  \item $D(\Phi_f \Phi_g) = D(fg) \cap D(g)$, and $\overline{\Phi_f \Phi_g} = \Phi_{fg}$
  \item if $g \in \Linftyphi$ then $\Phi_f + \Phi_g = \Phi_{f+g}$, and $\Phi_f \Phi_g = \Phi_{fg}$.
\end{enumerate}

\item
For $f \in \Lzerophi$, one has
\[
\Phi_f \in \lh \text{ if, and only if, } f \in \Linftyphi.
\]

\item
For $f \in \Lzerophi$, we have
\[
\spe \Phi_f = \bigcap_{A \in \mblsx} \overline{f(A)};
\]
where, $A$ runs over all $A \in \mblsx$ such that $E_A = I$ and $A \subset D(f)$.


\end{enumerate}

\end{thrm}












