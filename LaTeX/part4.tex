
\section{Extension Of Continuous Spectral Measures}

We now want to extend a given spectral measure for continuous functions, like the one we obtained
by the Gelfand-Neimark theorem, to 
measurable ones. To do so, we need to check whether the operators we attain
are bounded, or densely defined.
\begin{defi}[Spectral measure]
 
 Let $X$ be a compact space and let
 \[
  \Phi : \cf (X) \to \lh
 \]
be a map. $\Phi$ is called a \textit{spectral measure}, if its image 
$\A \coloneqq \Phi(\cf (X))$ is a commutative star-subalgebra of $\lh$ and 
$\Phi$ induces an isomorphism onto its image.
\end{defi}

\begin{rem}
 By isomorphism we mean that
 \begin{enumerate}
  \item $\Phi$ is an involutive algebrahomomorphism,
  \item $\Phi$ is a bijection onto A,
  \item $\Phi$ is an isometry: $\| \Phi f \| = \| f \|_\infty$.
 \end{enumerate}

\end{rem}

% Hier noch das Beispiel uebernehmen ?
\begin{expl}
Let $\m{}$ be a positive Radon measure on our compact space $X$, meaning a
continuous linear functional on $\cf (X)$, such that $\m{}(f) = 0$ implies 
that $f=0$. Set $\hs \coloneqq L^2(\mathfrak{m})$.
We define 
\begin{align*}
 \Phi : \cf(X) \to \mathscr{L}({\Ltwom}) \\
 f \mapsto (g \mapsto f\cdot g).
\end{align*}
$\Phi$ is a spectral measure. This claim is not obvious, and the proof
that $\Phi$ is an isometry requires a bit of work, but is omitted here.
It can be found in \cite{LesHaupt}. For all $g , h \in \hs$ define
\[
 \m{g, h} (f) \coloneqq \gen{g, \Phi_f h}.
\]
We then have that the map 
\[
 f \mapsto \m{g,h} (f)
\]
is a linear form on $\cf (X)$ for every pair $(g, h) \in \hs \times \hs$.
\end{expl}

Departing from this explicit example, we note that the previous construction
of a positive Radon measure, can be extended to any spectral measure. 
Let $\hs$ be any Hilbert space, and $\Phi$ be a spectral measure.
Then the mapping
\begin{align*}
  \m{g,h} : &\cf(X) \to \C \\
  &f \mapsto \gen{g, \Phi_f(h)}
\end{align*}
is a linear form on $\hs$.

\begin{thrm} \label{measureproperties}
For all $g, h,k \in \hs$, $\alpha, \beta \in \C$ and $f, \phi, \psi \in 
\cf(X)$, it holds that $\m{g,h}$ is a Radon measure on $X$ with the following 
properties:
\begin{enumerate}[{(}i{)}]
 \item $\| \m{g,h}\| \leq \|g\| \|h\|$
 \item $\m{\alpha g + h, \beta k} = \overline{\alpha}\beta \m{g,k} + 
 \beta \m{h,k}$
 \item $\overline{\mb}_{g,h} = \m{h,g}$, $\overline{\mb}_{g,h} : 
 f \mapsto \overline{\mb}_{g,h}(f) = \overline{\m{g,h}(\overline{f})}$
 \item $\m{g,g} \geq 0$
 \item $\m{\Phi_\phi g, \Phi_\psi h} = \overline{\phi} \psi \m{g,h}.$
\end{enumerate}

\end{thrm}

\begin{proof}
 \begin{enumerate}[(i)]
  \item \begin{align*}
         \| \m{g,h} \| &= \sup\limits_{\| f \| _\infty \leq 1} | \m{g,h}(f) | \\
			&= \sup\limits_{\| f \| _\infty \leq 1} | \gen{g, \Phi_f h} | \\
			&\leq \sup\limits_{\| f \| _\infty \leq 1} \| g \| \|\Phi_f \| \| h\| \\
			&= \|g\| \|h\|
        \end{align*}
  since $\|\Phi_f \| =\| f \|_\infty$.
 \item follows immediately from the linearity of $\gen{\cdot , \cdot}$
 
 \item $\overline{\mb}_{g,h}(f) = \overline{\gen{g, \overline{f}h}} = 
       \gen{\overline{f}h ,g} =	f\gen{h,g} = \gen{h, fg} = \m{h,g}(f).$
	
 \item Let $\phi \geq 0$. Because $\Phi$ is algebrahomomorphism we have 
	\[
	  \Phi_\phi = \Phi_{\sqrt{\phi}} \Phi_{\sqrt{\phi}} = 
	  \Phi_{\overline{\sqrt{\phi}}} \Phi_{\sqrt{\phi}} =
	  \Phi_{\sqrt{\phi}}\str \Phi_{\sqrt{\phi}}.
	\]
	Therefore $\m{g,g}(\phi) = \gen{g, \Phi_\phi g} = \gen{\Phi_{\sqrt{\phi}}g,
	\Phi_{\sqrt{\phi}}g} \geq 0. $
	
 \item	$\m{\Phi_\phi g, \Phi_\psi h} (f ) = \gen{\Phi_\phi g, \Phi_f 
	\Phi_\psi h} = \gen{g , \Phi_{\overline{\phi} \psi f} h } = \overline{\phi}
	\psi \m{g,h}(f)$ 

\end{enumerate}

\end{proof}

% Let $\A ^c $ denote the commutant of $\A$.
% 
% \begin{lem}
%  Let $S \in \lh$. $S \in \A ^c$ if and only if $\m{g, Sh} = \m{S \str g, h}$ 
%  for all $g, h \in \hs$. 
% \end{lem}
% 
% \begin{proof}
%  Follows from definition.
% \end{proof}
We now extend $\m{g,h}$ to measurable functions in the sense of \cite{PedAnaN}
Chapter 4.5.
\begin{defi}
 $N \subset X$ is called a $\Phi$-set of measure zero, or $\Phi$-null set, if
 $N$ is a null set of $| \m{g,h}|$ for all $g,h \in \hs$, that is 
 \[
  | \m{g, h} | (N) =0.
 \]
 Note that $| \m{g, h} |$ is a real valued Radon measure. Hence we can talk 
 about measurability of sets in the sense of Pedersen, found in \cite{PedAnaN}
 \marginparr{Kapitelzitierung aendern?}. \newline
 A function $f: X \to \C$ is called $\Phi$-measurable, if $f$ is 
 $|\m{g,h}|$-measurable for all $g,h \in \hs$. 
 Denote the set of all measurable functions  $\Lzerophi$. \newline \indent
  For technical 
 reasons, we restrict the previous definition
 to functions that do not take the value $\infty$ on a nonzero set. As it turns
out, the operators one gets from those functions are not even densely defined,
and hence of no interest to us. 

 \begin{gather*}
   \Lonephi \coloneqq \setl f \in \Lzerophi \bigm | f \in \Lone (\m{g,h}) 
   \text{ for all }g,h \in \hs \setr \\
   \Linftyphi \coloneqq \setl f \in \Lzerophi \bigm |  \|f \| _\infty 
   < \infty \setr\\
   \shortintertext{where, }
  \| f \| _\infty \coloneqq  \inf \setl \lambda > 0 \bigm |
    |f| \leq \lambda,  \phiae\setr.
 \end{gather*}
  By $\mblsx$ we denote the measurable subsets of $X$:
 \[
 \mblsx \coloneqq \setl A \subset X \bigm | 1_A \in \Lzerophi \setr.
 \]
 Let $f \in \Lzerophi$.
 \begin{align*}
 \dom(f) &\coloneqq \big\lbrace h \in \hs \bigm | f \in \Lone  (\m{g,h})
 \text{ for all } g
 \in \hs\\
 &~~~~~~~~\text{ and } g \mapsto \int f \dm{g, h} \text{ is continuous} 
 \big\rbrace\\
 &=  \big\lbrace h \in \hs \bigm | f \in \Lone (\m{g,h}) \text{ for all } g \in
 \hs \\
 &~~~~~~~~\text{ and } \exists k \in \hs \text{ such that } \int f \dm{g,h}
 = \gen{g, k} \big\rbrace
 \end{align*}
where the second equality is due to the Riesz representation theorem. The 
reader familiar with unbounded operators, will recognize the similarity
with the definition of the adjoint operator.
Using the second equality, we define 
\[
\Phi_f h \coloneqq k(h, f) = k \text{ for } h \in \dom(f) \eqqcolon
\dom(\Phi_f).
\]
In other words we have 
\[
\int f \dm{g,h} = \gen{g, \Phi_f h} = \gen{g, k}.
\]
By sesquilinearity of $\gen{\cdot, \cdot}$, $\Phi_f$ is linear as well.
\end{defi}


\begin{rem}

If we want to proof claims about $\Lzerophi$ it is enough to show them for 
$\cf(X)$, by since the expansion of measure is unique.
 
\end{rem}

\begin{lem} \label{lemlinm}
 
  For all $f \in \Lzerophi$ and $h \in \dom(f)$, $g\in \hs$ it holds that
 \[
 \m{g, \Phi_f h} = f\m{g,h}
 \]

\end{lem}

\begin{proof}
 By the remark, let $\phi \in \cf(X)$.
 \begin{align*}
   \m{g, \Phi_f h}(\phi) &= \gen{g, \Phi_\phi (\Phi_f h)} 
			  = \gen{\Phi_{\overline{\phi}} , \Phi_f h}
			  = \int f \dm{\Phi_{\overline{\phi}} g, h}\\
			  &= \int f \phi \dm{g,h}
			  = \left( f \m{g,h} \right)(\phi)
 \end{align*}

\end{proof}

We now want to know, whether the operator we obtain from our expanded measure
is bounded, and if not, is it densely defined. This will be done in the next
lemmata, which are going to be summarized in a theorem at the end of the
chapter.

\begin{lem}
  \label{maintheorem1}
 If $f \in \Linftyphi$, then $\dom(f) = \hs$, $ \Phi_f \in \lh$ and
 \[
 \| \Phi_f \| = \| f\|_\infty = \inf \setl \alpha > 0 \bigm |
 \alpha \geq |f|, \phiae\setr.
 \]

\end{lem}
\begin{proof}
   
   
   
   Let $f \in \Linftyphi$.
   We have to show that $\dom(f) = \hs$.
   Let $h \in \hs$. For $g \in \hs$ one gets:
   \begin{align*}
     \left | \int f \dm{g, h} \right | &\leq \int |f| \text{d}  |\m{g,h}| \\
			   &\leq \| f\| _\infty \int \text{d}  |\m{g,h}|\\
			   &= \| f\|_\infty \| \m{g,h} \|  \\
			   &\leq \| f\|_\infty \|h\| \|g\| 
\shortintertext{which implies}
g &\mapsto \int f \dm{g,h}
\end{align*}
is continuous and $f$ an element of $ \mscr{L}^1(\m{g,h})$.
The third equality follows from
\[
\|\m{g,h}\|=\sup_{\| \phi \|_\infty \leq 1} |\m{g,h}| =
\sup_{ \substack{| \phi | \leq 1\\ \phi \in \cf(X)}} |\m{g,h}|
 =  \int \text{d}  |\m{g,h}| = |\m{g,h}|(1) .
\]
This shows that the assignment $g \mapsto \int f \dm{g,h}$ defines a
continuous map.
\begin{align*}
  \| \Phi_f \| &= \sup_{\| g\|, \| h\| \leq 1}  
  \left| \gen{g, \Phi_f h} \right | \\
  &= \sup_{\| g\|, \| h\| \leq 1}  \left| \int f \dm{g,h} \right | \\
  &\leq \|f\|_\infty
\end{align*}

Other inequality, will be proved in Lemma \ref{maintheorem5}.
\end{proof}

\begin{rem}
 If $ f_n \to f \text{ in } \Linftyphi, $ then
\[
\Phi_f = \lim \Phi_{f_n} \text{ in } \lh.
\]
We can therefore define our spectral measure for all 
bounded measurable functions.
\end{rem}






\begin{lem}
  \label{maintheorem2}
Let $f \in \Lzerophi, (f_n)$ a net in $\Linftyphi$, and 
$\alpha, \beta \geq 0$ such that
 \begin{align*}
   | f_n | \leq \alpha |f| + \beta \text{ for all } n \\
   f_n \to f \phiae.
 \end{align*}
Then the following statements about $h \in \hs$ are equivalent:
\begin{enumerate}[(i)]
 \item $h \in \dom(f)$
 \item $\int \str | f| ^2 \dm{h,h} < \infty$
 \item $(\Phi_{f_n}h)$ converges in $ \hs$
\end{enumerate}
One then has
\begin{align*}
  \| \Phi_f h \| ^2 &= \int |f|^2 \dm{h, h } \text{ and }\\
  \Phi_{f }h &= \lim \Phi_{f_n} h.
\end{align*}
For example the net $f_n = 1_{\mc{A}_n}$, where $\mc{A}_n = 
\setl x \in X \bigm | | f(x) | \leq n \setr$.
\end{lem}

\begin{proof}
 $(i) \Rightarrow (ii)$:
 
 Let $h \in \dom(f)$.
 \begin{align*}
   \infty > \|\Phi_f h\| ^2 &= \gen{ \Phi_f , \Phi_f h} \\
   &= \int f \dm{\Phi_f h, h} \\
   &= \int f \overline{f} \dm{h,h} \\
   &= \int | f | ^2 \dm{h,h}.
 \end{align*}
  $(ii) \Rightarrow (iii)$:
  
  Let $h \in \mscr{L}^2(\m{h,h})$. Then for $g \in \hs$
  \begin{align*}
    \gen{g, (\Phi_{f_m} - \Phi_{f_n})h} &= \gen{g, \Phi_{f_m}} 
    - \gen{g, \Phi_{f_n}}\\
					&=\int f_m \dm{g,h} - \int f_n \dm{g,h} \\
					&= \int (f_m - f_n) \dm{g,h} \\
					&= \gen{g, \Phi_{(f_m - f_n)} h }
    \intertext{Since $g$ was arbitrary, we get 
      $(\Phi_{f_m} - \Phi_{f_n})h= \Phi_{(f_m - f_n)}h$}
    \|(\Phi_{f_m} - \Phi_{f_n}) h \|^2  &= \| \Phi_{(f_m - f_n)} h \| ^2\\
					&= \int |f_m - f_n|^2 \dm{h,h}.
  \end{align*}
where the last equality holds, because $h \in \dom(f_m - f_n)$ and thus $f \in \Ltwophi$. 
By our assumption, we have $f_m -f_n \to 0 \phiae$ Thus
\[
| f_m - f_n |^2 \leq (2(\alpha |f| + \beta))^2 \in \mscr{L}^1(\m{h,h}),
\]
because $\mscr{L}^2(\m{h,h}) \subset \mscr{L}^1(\m{h,h}) $
by compactness of $X$, and
$\cf(X) \in \mscr{L}^1(\m{h,h})$ by definition.
Therefore Lebegues theorem about dominated convergence yields the claim.

$(iii) \Rightarrow (ii)$:\newline
Using Fatous lemma, and again as above
$\| \Phi_{f_n} h \|^2 = \int | f_n | ^2 \dm{h,h} $, we get 
\begin{align*}
  \infty & > \| \lim_{n \to \infty} \Phi_{f_n} h \| ^2 = \lim_{n \to \infty}
  \| \Phi_{f_n} h \| ^2 =
  \lim_{n \to \infty} \int | f_n | ^2 \dm{h,h} \\
  & \geq \int \str \liminf_{n \to \infty} |f_n | ^2 \dm{h,h} = \int \str |f|^2 \dm{h,h}
\end{align*}

$(ii) \Rightarrow (i)$: \newline
 By Radon-Nikodym, there exists a Borel-measurable function 
 $\phi : X \to \C$, $|\phi| = 1$, such that 
 \[
 | \m{g,h} | = \phi \m{g,h}.
 \]
 Now define $\tilde{f} \coloneqq \phi |f|$, $\tilde{f_n} \coloneqq
 \phi |f_n|$. Thus 
 \[
 \int \str |\tilde{f} |^2 \dm{h,h} = \int \str |f|^2 \dm{h,h} < \infty.
 \]
 By $(ii) \Rightarrow (iii)$, $\lim \Phi_{\tilde{f_n}}$ exists. Hence, for $g \in \hs$
 
 \begin{align*}
   \infty &> \|g\|^2 \| \lim_{n \to \infty} \Phi_{\tilde{f}_n}h \|^2 \geq
   \left|\gen{g, \lim_{n \to \infty} \Phi_{\tilde{f_n}} h} \right| =
   \left|\lim_{n \to \infty} \int \tilde{f}_n \dm{g,h} \right| \\
	  &\geq \lim_{n \to \infty} \int |f_n| \text{d}  |\m{g,h}| 
	  \geq \int \str \liminf_{n \to \infty} |f_n| \text{d}  |\m{g,h}| 
	  = \int \str |f| \text{d}  |\m{g,h}|,
 \end{align*}
 which implies that $f \in \mscr{L}^1(\m{g,h})$. 
 So once more by Lebegues theorem
 \[
 \gen{g, \lim_{n \to \infty} \Phi_{f_n} h } = 
 \lim_{n \to \infty} \int f_n \dm{g,h} = \int f \dm{g,h}
 =\gen{g, \Phi_f h},
 \]
which means, $h \in \dom(f)$. 


\end{proof}




\begin{lem}
  \label{maintheorem3}
 For each $f \in \Lzerophi$, $\Phi_f$ is a normal Operator, and we have
\[
\Phi_f \str = \Phi_{\overline{f}}. 
\]
If $f$ is real valued, then $\Phi_f$ is self-adjoint, and $\Phi_{1_A} 
\eqqcolon E_A$ for $ A \in \mblsx$ is an orthogonal projection. 
\end{lem}


\begin{proof}

First we show that $\Phi_f$ is densely defined. \newline
We claim $E_{\mc{A}_n}(\hs) \subset \dom(f)$, where $\mc{A}_n = 
\setl x \in X \bigm | | f(x) | \leq n \setr$.
Let $h \in \hs$. By Lemma \ref{maintheorem2} the claim is
equivalent to 
\[
\int \str |f|^2 \dm{E_{\mc{A}_n}h, E_{\mc{A}_n}h} < \infty.
\]
We have 
\begin{align*}
  \int \str |f|^2 \dm{E_{\mc{A}_n}h, E_{\mc{A}_n}h} = 
  \int \str \overline{1}_{\mc{A}_n} 1_{\mc{A}_n} \dm{h,h} \leq
  n^2 \int \dm{h,h} \leq n^2 \| h\|^2 < \infty.
\end{align*}

For $h \in \hs$ we have $h = \lim_{n \to \infty} E_{\mc{A}_n} h$. 
Then $|1_{\mc{A}_n}| \leq 1$, and we have
$1_{\mc{A}_n} \to 1 $ pointwise $\phiae$
Let $h \in \hs = \dom(1)$. By Lemma \ref{maintheorem2} we have
\[
h = \Phi_1 h = \lim_{n \to \infty} \Phi_{1_{\mc{A}_n}} =
\lim_{n \to \infty} E_{\mc{A}_n} h , 
\]
which gives $\dom(f) \subset \hs$ is dense, as $E_{\mc{A}_n} h \in \dom(f)$ by
the claim. 

Now we claim $ \Phi_{\overline{f}} \subset \Phi_f \str$.\newline
Let  $g,h \in \dom(f) = \dom(\overline{f})$. Using Theorem \ref{measureproperties} $
(iii)$, one has
\begin{align*}
  \gen{g, \Phi_f h} &= \int f \dm{g,h} 
  = \overline{ \int \overline{f} \text{d}\overline{\mf{m}_{g,h}}} 
  = \overline{ \gen{h , \Phi_{\overline{f}} g }} 
  = \gen{\Phi_{\overline{f} }g , h}
\end{align*}
So, $g \in \dom(\Phi_f \str)$ and $\Phi_f \str g = \Phi_{\overline{f}} g$.

On the other hand, to show that  $\Phi_{\overline{f}} \supset \Phi_f \str$
let $g\in \dom(\Phi_f \str)$.
By Lemma \ref{maintheorem2}, we only have to show $\Phi_{\overline{f}_n}g$
converges in $\hs$, for some net satisfyig the conditions of Lemma
\ref{maintheorem2}. As a net, we take $f_n = 1_{\mc{A}_n}$,
where $\mc{A}_n = \setl x \in X \bigm | | f(x) | \leq n \setr$.
Let $h \in \hs$.

\begin{align*}
\gen{ \Phi_{\overline{f}_n} g, h }& = \int \dm{\Phi_{\overline{f}_n} g, h}
= \int f_n \dm{g,h} 
= \int f \dm{g, \Phi_{1_{\mc{A}_n }} h }
= \int \dm{g, \Phi_f E_{\mc{A}_n} h} \\
&= \gen{g, \Phi_1 \Phi_f E_{\mc{A}_n }h }
= \gen{g, \Phi_f E_{\mc{A}_n} h} 
= \gen{\Phi_f \str g, E_{\mc{A}_n} h}
= \int 1_{\mc{A}_n} \dm{\Phi_f \str g, h}\\
&= \int \dm{E_{\mc{A}_n} \Phi_f \str g, h} = \gen{E_{\mc{A}_n}  \Phi_f \str g, h}
\end{align*}

\[
\Phi_{\overline{f}_n} g = E_{\mc{A}_n} \Phi_f \str g 
\xrightarrow{n \to \infty} \Phi_f \str g \text{ , so  }
\Phi_{\overline{f}} = \Phi_f \str.
\]


Now we claim $\Phi_f$ is a normal element. To show
\[
\dom(\overline{f}) = \dom(\Phi_f \str) = \dom(\Phi_{\overline{f}}) = \dom(f),
\]
 which is already proven. Also already proven is now
\[
 \| \Phi_f \str h \| = \| \Phi_f h \|,
\]
for example, by the norm formula of Lemma \ref{maintheorem2}.
Thus $\dom(\Phi_f \str) = \dom(\Phi_{\overline{f}})$ and 
$\| \Phi_f \str h \| = \| \Phi_f h \|$, which proves that $\Phi_f$ is normal.
To show that this implies the usual definition of a normal operator,
is left as an easy excercise using the polarization identity.

If now $f$ is real valued, we have that $f = \overline{f}$, which gives 
the selfadjointness of $\Phi_f$. Furthermore $E_\mc{A} \str = E_\mc{A}$.

\end{proof}

\begin{lem}
  \leavevmode
  \label{maintheorem3+}
 \begin{enumerate}
  \item $f \in \Linftyphi,$ 
  \[
   f \geq 0 \phiae \Rightarrow \Phi_f \geq 0
  \]

  \item $\mc{A} \in \mblsx,$
  \[
  E_\mc{A} =0  \Leftrightarrow \mc{A} ~\Phi\text{-null set}
  \]
  \item $U \in \mblsx\text{ open,}$ 
  \[
   U \neq \varnothing \Rightarrow E_U \neq 0
  \]

 \end{enumerate}

\end{lem}



\begin{proof}
  Left as an exercise.
%   \leavevmode
%  \begin{enumerate}
%    \item By Lemma \ref{measureproperties} (iv), $\m{h,h}$ is a positive
%    measure. Thus 
%    \[
%       \gen{h, \Phi_f h } = \int f \dm{h,h} \geq 0
%    \]
% 
%    \item   \begin{align*}
%      E_\mc{A} =0 & \Leftrightarrow E_\mc{A} h =0 \text{ for all } h \in \hs \\
%      &\Leftrightarrow \gen{g, E_\mc{A} h } =0 \text{ for all } h \in \hs \\
%      &\Leftrightarrow \int 1_\mc{A} \dm{g,h} =0 \text{ for all } h \in \hs \\
%      &\Leftrightarrow \mc{A} \text{ is a } \Phi\text{-null set}.
%    \end{align*}
%    \marginparr{Fuer alle aendern?}
%    \item 
%    If $U$ open $\neq \varnothing$, then by Tieze-Uryson, there exists a
%    positive fucntion $f : X \to \R_+$, such that $f\neq 0$, and 
%    $f\restriction U^c = 0$. Without loss of generality, we may assume 
%    that $0< f \leq 1_{U}$. Then by $(i)$, $E_{U}= \Phi_{1_{U}} \geq \Phi_f > 0$.
%  \end{enumerate}

\end{proof}

\begin{lem} 
\label{maintheorem4}
  
 For each $\varphi, \psi \in \Lzerophi$, $\alpha \in \C$, we have

\begin{enumerate}[(i)]
  \item $\Phi_{\alpha \varphi} = \alpha \Phi_\varphi$
  \item $\dom(\Phi_\varphi \Phi_\psi) = \dom(\varphi \psi) \cap \dom(\psi)$, and 
  $\overline{\Phi_\varphi \Phi_\psi} = \Phi_{\varphi \psi}$
  \item $ \overline{\Phi_\varphi + \Phi_\psi} = \Phi_{\varphi + \psi}$
  \item $\psi \in \Linftyphi \Rightarrow \Phi_\varphi + \Phi_\psi =
  \Phi_{\varphi+\psi}$, and $\Phi_\varphi \Phi_\psi = \Phi_{\varphi \psi}$.
%   = \Phi_{\psi \varphi} = \Phi_\psi \Phi_\varphi$.
\end{enumerate}

  
\end{lem}
\marginparr{Proof wirklich durch enumerate?}
\begin{proof}
  \begin{enumerate}[(i)]
  \item 
  \begin{align*}
  \dom(\alpha \varphi) = \setl h \in \hs \bigm | \int \str |\alpha \varphi|^2
  \dm{h,h} < \infty  \setr = \dom(\varphi)   \shortintertext{ and,}
  \gen{g, \Phi_{\alpha \varphi} h } = \int \alpha \varphi \dm{g,h} = 
  \alpha \int \varphi \dm{g,h} = \gen{\psi, \alpha \Phi_\varphi h}
  \end{align*}
  for all $g \in \hs$. 

\item 
  \[
  h \in \dom(\Phi_\varphi \Phi_\psi ) \Leftrightarrow h \in \dom(\psi) \text{ and }
  \Phi_\psi h  \in \dom(\varphi).
  \]
  By Lemma \ref{lemlinm}, for $h \in \dom(\psi)$, $\m{g,\Phi_\psi h} = \psi \m{g,h}$
  \begin{align*}
    \int \str |\varphi| \text{d}|\m{g, \Phi_\psi h}| &= \int \str |\varphi \psi|
    \text{d}|\m{g,h}|,
    \shortintertext{so}
    \int \str |\varphi| \text{d}|\m{g, \Phi_\psi h}| < \infty &\Leftrightarrow
  \int \str |\varphi \psi| \text{d}|\m{g,h}| < \infty
  \shortintertext{and }
  g \mapsto \int \str \varphi \dm{g, \Phi_\psi h} \text{ is continuous } 
  &\Leftrightarrow
  g \mapsto \int \str \varphi \psi \dm{g,h} \text{ is continuous}.
  \end{align*}
But the last statement, reformulates to 

\begin{gather*}
   h \in \dom(\psi) \text{ and } \Phi_\psi h \in \dom(\varphi) \Leftrightarrow
   h \in \dom(\psi) \text{ and } h \in \dom(\varphi \psi).\\
  \gen{g, \Phi_\varphi \Phi_\psi h } = \int \varphi \dm{g, \Phi_\psi h} = 
  \int \varphi \psi \dm{g,h} = \gen{g, \Phi_{\varphi \psi} h }
\end{gather*}

Thus $\Phi_\varphi \Phi_\psi \subset \Phi_{\varphi \psi}$. The prove of 
$\overline{\Phi_\varphi \Phi_\psi} = \Phi_{\varphi \psi}$ is analogous to the
prove of $ \overline{\Phi_\varphi + \Phi_\psi} = \Phi_{\varphi + \psi}$ and
will be omitted.

  
  \item 
   
  To show
  \[
   \dom(\varphi) \cap \dom(\psi) \subset \dom(\varphi + \psi).
  \]
  Let $h \in \dom(\varphi) \cap \dom(\psi)$. By Lemma \ref{maintheorem2}, 
  \begin{gather*}
    \int \str |\varphi|^2 \dm{h,h},~ \int \str |\psi|^2 \dm{h,h} < \infty ,
    \shortintertext{and by Minkowskys
  inequality}
    \left( \int \str | \varphi+\psi|^2 \dm{h,h} \right)^{\frac{1}{2}} \leq
    \left( \int \str | \varphi|^2 \dm{h,h} \right)^{\frac{1}{2}} +
    \left( \int \str |\psi|^2 \dm{h,h} \right)^{\frac{1}{2}}.
  \end{gather*}
  Thus $h \in \dom(\varphi+\psi)$. Furthermore, for  $g \in \hs$
  \[
  \gen{g, (\Phi_\varphi + \Phi_\psi)h }= \int\! \varphi \dm{g,h} +
  \int \! \psi \dm{g,h}=
  \int \! (\varphi+\psi) \dm{g,h} = \gen{g, \Phi_{\varphi+\psi} h},
  \]
  and thus $\Phi_\varphi + \Phi_\psi \subset \Phi_{\varphi+\psi}$.
  The rest of the proof will follow after part $(iv)$.
  
  

\item 
$\psi \in \Linftyphi$ implies that $\dom(\psi)$ is already the whole space.
Thus
\begin{align*}
 \dom(\Phi_\varphi \Phi_\psi) = \dom(\varphi \psi)\cap \hs = \dom(\varphi \psi),
 \shortintertext{and}
 \dom(\Phi_\varphi + \Phi_\psi) = \dom(\varphi) \cap \hs = \dom(\varphi).
%  \end{align*}  
%  \begin{align*}
%  \intertext{For $\dom(\psi \varphi)$ we calculate}
%   \int \str | \psi \varphi |^2 \dm{h,h} \leq 
%   \| \psi\|_\infty ^2 \int \str |\varphi|^2 \dm{h,h}\\
%   \intertext{and thus get using $(i)$,}
%   \dom(\varphi) \subset \dom(\psi \varphi) \Rightarrow \dom(\Phi_\varphi \Phi_\psi)
%   = \dom(\varphi).
  \shortintertext{Therefore, we have proven}
  \Phi_\varphi \Phi_\psi = \Phi_{\varphi \psi} \text{ and }
 \Phi_\varphi \Phi_\psi = \Phi_{\varphi + \psi}.
\end{align*}
In particular, 
\[
\mc{A} \in \mscr{E}(\Phi) \Rightarrow E_{\mc{A}} \text{ is a projection.}
\]
  
\end{enumerate}
\noindent Rest of $(ii)$
\begin{adjustwidth}{0.85cm}{}
  For $ \overline{\Phi_\varphi + \Phi_\psi} = \Phi_{\varphi + \psi}$, 
  we need to show that for
  $h \in \dom(\varphi + \psi)$, there exists a net 
  $(h_n) \in \dom(\varphi)\cap \dom(\psi)$, such that
  $\lim h_n = h$, and $\lim (\Phi_\varphi h_n +\Phi_\psi h_n) =
  \Phi_{\varphi + \psi}h$. Set
  $\mc{A}_n = \setl x \in X \bigm | |\varphi(x)| + |\psi(x)| \leq n\setr$.
  By Lemma \ref{maintheorem2}, we have
  
  \begin{gather*}
    E_{\mc{A}_n}(\hs) \subset \dom(\varphi)\cap \dom(\psi)
    \shortintertext{and}
    \cup \mc{A}_n = X,~\mc{A}_n \subset \mc{A}_{n+1}.
    \shortintertext{And thus}
    \lim E_{\mc{A}_n} = \id  
  \end{gather*}

  For all $h \in \dom(\varphi + \psi)$, one has
  \begin{align*}
  h &= \lim E_{\mc{A}_n} h \eqqcolon \lim h_n,
  \intertext{and using $(iv)$ combined with the fact that $E_{\mc{A}_n} $
    is bounded, we get}
  \Phi_{\varphi+\psi} h  &=\lim E_{\mc{A}_n} (\Phi_{\varphi+\psi})h \\
  &= \lim \Phi_{1_{\mc{A}_n} (\varphi+\psi)} h \\
  &= \lim \Phi_{(\varphi + \psi) 1_{\mc{A}_n}} h \\
  &= \lim \Phi_{\varphi+\psi} E_{\mc{A}_n} h \\
  &= \lim \Phi_{\varphi+\psi} h_n = \lim (\Phi_\varphi h_n + \Phi_\psi h_n)
  \end{align*}
\end{adjustwidth}
\end{proof}

\begin{lem} \label{maintheorem5}
  For $f \in \Lzerophi$, one has
\[
\Phi_f \in \lh \Leftrightarrow f \in \Linftyphi.
\]

 
\end{lem}

\begin{proof}
 "$\impliedby$" already proven in Lemma \ref{maintheorem1}.
 
 For the other direction we proof $\| \Phi_f \| \geq \| f\|_\infty$.
 Let $\lambda < \| f\|_\infty$. Then $\mc{A}_\lambda \coloneqq 
 \setl |f| \geq \lambda \setr$ is not a $\Phi$-null set. By the polarization
 identity, there exists a $h \in \hs$, such that $\mc{A}_\lambda$ is not a
 $\m{h,h}$-null set. We have
 \begin{align*}
   \mc{A}_\lambda= \bigcup_{\substack{\mu \in \Q \\ \mu > \lambda}}
   \setl \lambda \leq | f| \leq \mu \setr
   \shortintertext{which gives}
   0 \neq \m{h,h}(\mc{A}_\lambda) = 
   \sup_{\substack{\mu \in \Q \\ \mu > \lambda}} \m{h,h}(
   \setl \lambda \leq |f| \leq \mu \setr ).
 \end{align*}
  Therefore, there exists a $\mu > \lambda$, such that
  \[
  \m{h,h}( \setl \lambda \leq |f| \leq \mu \setr) \eqqcolon \m{h,h}(B) > 0.
  \]
  We then have that $E_B h \in \dom(f)$, since f is bounded on $B$. Note that
  this is a priori not true for $\mc{A}_{\lambda}$.
  \begin{align*}
    \| \Phi_f E_B  h \|^2 &= \int |f|^2 \dm{E_B h,E_B h} \\
    &= \int_B |f|^2 \dm{h,h}\\
    &\geq \lambda ^2 \m{h,h}(B) \\
    &= \lambda ^2 \int \overline{1}_B 1_B \dm{h,h}\\
    &= \lambda ^2 \int \dm{E_B h, E_B h} = \lambda ^2 \| E_B h \|^2 
  \end{align*}
  For $\tilde{h} \coloneqq \frac{E_B h }{\| E_B h\|} \in \dom(f)$, we have
  $\| \Phi_f \tilde{h} \| \geq \lambda$. Since $\| \tilde{h} \| = 1$,
  $\| \Phi_f\| \geq \lambda.$
\end{proof}


\begin{lem} \label{maintheorem6}
  For $f \in \Lzerophi$ \unsure{ Umformattieren.
  \vspace{3cm}}
\begin{align*}
  \Phi_f \text{ is invertible} \text{ if and only if } \setl f = 0 \setr
  \text{ is a }   \Phi \text{-null set}  \text{ and } \frac{1}{f}
  \in \Linftyphi
\end{align*}

One then has
\[
\Phi_f \inv = \Phi_{\frac{1}{f}}
\]

\end{lem}

\begin{proof}
 $\Leftarrow$: 
 By the previous Lemma $\Phi_\frac{1}{f} \in \lh$, and by Lemma 
 \ref{maintheorem4}
 \[
 \Phi_\frac{1}{f} \Phi_f \subset \Phi_{\frac{1}{f} f} = \Phi_1 = \id =
 \Phi_{f \frac{1}{f}} = \Phi_f \Phi_\frac{1}{f}.
 \]
Hence, $\Phi_f$ is invertible.

$\Rightarrow$:
Let $h \in E_{\setl f = 0\setr}(\hs)$. Since $E_{\setl f = 0\setr}$ is a
projection, $E_{\setl f = 0\setr} h = h$. Thus
\[
  \Phi_f h = \Phi_f E_{\setl f = 0\setr} h = \Phi_{f \cdot 
    1_{\setl f = 0\setr}} h = \Phi_0 h = 0.
\]
Since $\Phi_f $ is invertible, $h =0$ and $E_{\setl f = 0\setr} =0$ are
immediate consequences. Therefore $\setl f = 0\setr$ is a $\Phi$-null set. 

It remains to show that $ \frac{1}{f} \in \Linftyphi$. We have
\[
 \Phi_f \cdot \Phi_\frac{1}{f} \subset \Phi_1 = I.
\]
On $\dom(\Phi_f \cdot \Phi_\frac{1}{f}) $, it holds that $\Phi_\frac{1}{f}
= \Phi_f \inv$ .
Since $\Phi_f$ is invertible, it is surjective. We claim $ \hs = \Phi_f(\dom(f))
\subset \dom(\Phi_\frac{1}{f})$

\begin{align*}
  \int \left| \frac{1}{f} \right | \text{d}|\m{g,\Phi_f h}| &= \sup_{| \phi |
    \leq \left | \frac{1}{f} \right| } \left | \int \phi \dm{g,\Phi_f h}
  \right| \\&= \sup_{| \phi | \leq \left| \frac{1}{f} \right |}
  \left| \int \phi f \dm{g,h} \right| \leq \int \text{d}|\m{g,h}| \leq
  \| h \| \|g\|.
\end{align*}
Thus $\dom(\Phi_\frac{1}{f}) = \hs$, which implies $\frac{1}{f} \in \Linftyphi$,
by Lemma \ref{maintheorem5}.
\end{proof}

\begin{lem} \label{maintheorem7}
 Let $f \in \Lzerophi$. Then 
 \[
 \spe \Phi_f = \bigcap_{E_\mc{A} = I} \overline{f(\mc{A})}
 \]

\end{lem}

\begin{proof}
 " $\subset$ "
 Fix $\lambda \in \spe \Phi_f$, $\mc{A} \subset X$ such that $E_\mc{A} = I$.
 We claim that $\lambda \in \overline{f(\mc{A})}$. By Lemmata \ref{maintheorem4}
 and \ref{maintheorem6}, $\Phi_f - \lambda I= \Phi_{f - \lambda}$ is 
 not invertible impies that either
\begin{align*}
  \left.  a\right)& \setl f- \lambda = 0 \setr \text{ is not a }\Phi
  \text{-null set}.\\
  \left.  b\right)& \setl f - \lambda =0 \setr \text{ is a }\Phi\text{-null set,
    but } \frac{1}{f- \lambda} \notin \Linftyphi.
\end{align*}

Suppose $a)$ holds. Since $\mc{A}\complement$ is a $\Phi\text{-null set}$,
\[
\mc{A} \cap \setl f= \lambda \setr \neq \varnothing .
\]

This means there exists a $x \in \mc{A}$ such that $f(x) = \lambda$, that is
\[
\lambda \in f(\mc{A}).
\]

Suppose now that $b)$ holds. By Lemma \ref{maintheorem6} $\frac{1}{f- \lambda}
\notin \Linftyphi$ implies that $\frac{1}{f- \lambda}$ is unbounded on
$\mc{A} \backslash \setl f= \lambda \setr$. Thus, there exists a sequence
$(x_n) \in \mc{A} \backslash \setl f= \lambda \setr$, such that
\[
 \lim | f(x_n) - \lambda| =0,
\]
which impies that
\[
\lambda = \lim f(x_n) \in \overline{f(\mc{A})}.
\]

"$\supset$"
Fix $\lambda \notin \spe (\Phi_f)$. We have to show that there exists a set
$\mc{A}_0$, $E_{\mc{A}_0} = I$ and $\lambda \notin \overline{f(\mc{A}_0)}.$
By Lemma \ref{maintheorem6}, $\Phi_f - \lambda I= \Phi_{f- \lambda}$ is 
invertible implies that $ \setl f = \lambda \setr  \text{ is a }   \Phi
\text{-null set,} \text{ and } \frac{1}{f - \lambda }  \in \Linftyphi$. Thus with 
$M \coloneqq \| \frac{1}{f - \lambda} \|_\infty$,
$\setl | f- \lambda | < \frac{1}{M} \setr$ is a $\Phi$-null set. Therefore,
we can set $\mc{A}_0 \coloneqq \setl | f- \lambda | \geq \frac{1}{M} \setr $.
Then $E_{\mc{A}_0} = I$ and $d(\lambda, \overline{f(\mc{A}_0)}) \geq
\frac{1}{M}$, that is 
\[
\lambda \notin \overline{f(\mc{A}_0)}.
\]

  
\end{proof}


Summing up the previous lemmata, we get our

\begin{thrm}[Main theorem] \label{maintheorem}
  \leavevmode
  \begin{enumerate}
   \item   

 If $f \in \Linftyphi$, then $\dom(f) = \hs$, $ \Phi_f \in \lh$ and
 \[
 \| \Phi_f \| = \| f\|_\infty = \inf \setl \alpha > 0 \bigm | \alpha \geq |f|, \phiae\setr.
 \]

 
\item 
Let $f \in \Lzerophi, (f_n)$ a net in $\Linftyphi$, and $\alpha, \beta \geq 0$ such that
 \begin{align*}
   | f_n | \leq \alpha |f| + \beta \text{ for all } n \\
   f_n \to f \phiae.
 \end{align*}
Then the following statements about $h \in \hs$ are equivalent:
\begin{enumerate}[(i)]
 \item $h \in \dom(f)$
 \item $\int \str | f| ^2 \dm{h,h} < \infty$
 \item $(\Phi_{f_n}h)$ converges in $ \hs$
\end{enumerate}
One then has
\begin{align*}
  \| \Phi_f h \| ^2 &= \int |f|^2 \dm{h, h } \text{ and }\\
  \Phi_{f }h &= \lim \Phi_{f_n} h.
\end{align*}
For example, one could take the net $f_n = 1_{\mc{A}_n}$,  where
$\mc{A}_n = \setl x \in X \bigm | | f(x) | \leq n \setr$.

\item
For each $f \in \Lzerophi$, $\Phi_f$ is a normal Operator, and we have
\[
 \Phi_f \str = \Phi_f. 
\]
If $f$ is real valued, then $\Phi_f$ is self-adjoint, and $\Phi_{1_A} 
\eqqcolon E_A$ for $ A \in \mblsx$ is an orthogonal projection. 
\item
\begin{enumerate}[(i)]
  \item $f \in \Linftyphi,$ 
  \[
   f \geq 0 \phiae \Rightarrow \Phi_f \geq 0
  \]

  \item $\mc{A} \in \mblsx,$
  \[
  E_\mc{A} =0  \Leftrightarrow \mc{A} ~\Phi\text{-null set}
  \]
  \item $U \in \mblsx\text{ open,}$ 
  \[
   U \neq \varnothing \Rightarrow E_U \neq 0
  \]

 \end{enumerate}


\item 
For each $f, g \in \Lzerophi$, $\alpha \in \C$, we have

\begin{enumerate}[(i)]
  \item $\Phi_{\alpha f} = \alpha \Phif$
  \item $ \overline{\Phif + \Phi_g} = \Phi_{f + g}$
  \item $\dom(\Phi_f \Phi_g) = \dom(fg) \cap \dom(g)$, and $\overline{\Phi_f \Phi_g}
  = \Phi_{fg}$
  \item if $g \in \Linftyphi$ then $\Phi_f + \Phi_g = \Phi_{f+g}$, and 
  $\Phi_f \Phi_g = \Phi_{fg}$.
\end{enumerate}

\item
For $f \in \Lzerophi$, one has
\[
\Phi_f \in \lh \text{ if, and only if, } f \in \Linftyphi.
\]

\item
For $f \in \Lzerophi$, we have
\[
\spe \Phi_f = \bigcap_{A \in \mblsx} \overline{f(A)};
\]
where, $A$ runs over all $A \in \mblsx$ such that $E_A = I$ and 
$A \subset \dom(f)$. 


\end{enumerate}

\end{thrm}












