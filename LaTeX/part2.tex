
\section{First things first}
We want to construct a most general spectral theorem for linear opeartors between Hilbert spaces. To achieve said goal, we need some auxilliary results, derived by Gelfand's theorem. For each normal operator $T \in \lh$, we will consider the associated normal $^*$-subalgebra 
\[
 \A(T) \coloneqq \gen{I, A, B, B^*},~ A \coloneqq \frac{1}{1 + T \str T}, ~ B \coloneqq TA
\]
which is well defined, since $1 + T \str T$ is invertible. Gelfand gives us a somewhat o.k. understanding of bounded linear operators, so we might hope to get information about $T$ if we use our understanding of $A$ and $B$, and reconstruct $T$ via $T = BA\inv$.
\newline First some corollaries from the Gelfand representation theorem
\begin{prop}
 Let $\A \subset \B$ be a unitary $^*$-subalgebra of $\B$. If $A \in \A$, $A \in \GL(\B)$, then $A \in \GL(\A)$. In other words 
\[
  \forall A \in \A: \spe_{\A}(A) = \spe_{\B}(A).
\]

\end{prop}
\begin{proof}
 First assume $A = A^* $. We have $\spe_\A(A) \subset \R$, implying that
 \[
  \forall \lambda \neq 0: (A + i \lambda I)\inv \in \A \subset \B.
 \]
 Let $A$ is invertible in $\B$. Since 
 \[
  \lim_{\lambda \to 0} (A + i \lambda I) = A,
 \]
 by continuity of the inverse map, we get
 \[
  \lim_{\lambda \to 0} (A + i \lambda I) \inv =A \inv \in \B.
 \]
Because for all $\lambda \neq 0$ we get $ (A + i \lambda I)\inv \in \A$, and $A$ is closed in $B$, the statement  holds for normal $A$. 

For general in $\B$ invertible, $A \in \A$, one has the normal element $A\str A \in \B$, with inverse $(A\str A)\inv = A\inv (A\inv)^*$. Since $\A$ is a $^*$-algebra, $A\str A \in \A$, which implies that $A$ is left-invertible in $\A$ with inverse $(A\str A)\inv A^*$. Using the same argument with the normal element $AA^*$, one gets the right-invertibility of $A$. Now, by basic group theory, $A$ is invertible, and the inverses coincide.
\end{proof}

\begin{cor}
 Let $\A$ be a $^*$-subalgebra of $\lh$, $T \in \A$. Then 
\[
 \spe_\A (T) = \spe_{\lh}(T) = \spe(T).
\]
\end{cor}

\begin{prop}[Functional calculus for normal elements]
 Let $\B$ be a $^*$-subalgebra with unit, and $A \in \B$ a normal element. Then the normal star-subalgebra generated by $A$ is isomorphic to $\cf (\spe A)$, where $\spe(A) \cong \spe(\A)$ via the Gelfandtransform.
\end{prop}

\begin{proof}
 First, we show that $\G_A : \spe(\A) \to \C$ is injective. If $\chi_1 , \chi_2 \in \spe(\A)$, $\G_A(\chi_1) = \G_A(\chi_2)=\chi_2(A)=\chi_1(A)$, then also $\chi_1(A^*)=\chi_2(A^*)$. Since $\chi_1(I)=\chi_2(I)=1$, we see that $\chi_1 = \chi_2$ on all polynomials in $A$, $A^*$. Because $\chi_1$, $\chi_2$ are continous, they coincide on $\A$ and $\G_A(\A)=\spe(a)$ by Gelfand. $\G_A$ is a continous bijection from $\spe(\A)$ to $\spe(A)$, $\spe(\A)$ is compact and therefore
\[
 \G_A \text{ is a homeomorphism from } \spe(\A) \text{ to } \spe(A).
\]
By Gelfand--Neimark 
\[
 \G : \A \to \cf(\spe \A)
\]
is an isomorphism. We get the following commutative diagram, which yields the result
\[
\xymatrix{
 \A \ar[rrr]^{{B \mapsto \G_B}} \ar[drrr]_{B \mapsto \G_B \circ \G_A \inv~~~} &&& \cf(\spe(\A)) \ar[d]^{\G_B \mapsto \G_B \circ \G_A \inv} \\
   &&&  \cf(\spe(A))}
\]
\marginparr{irgendwo mit tikz compilieren oder so lassen?}



\end{proof}

\begin{rem}
  Let $\Phi : \cf (\spe A) \to \A$ be the inverse of the isomorphism from the previous theorem defnied by $B \mapsto \G_B \circ \G_A \inv$.
 
  For $f \in \cf(\spe A)$ get 
  \[
   \Phi (f) = \G \inv (f \circ \G_A)).
  \]
  Thus, one gets back the generators of $\A$ via
   \begin{align*}
      \Phi (1_{\spe(A)})   		&= \G \inv (1 _{\spe(A)} \circ \G_A ) \\
				    &= \G \inv (1_{\spe(\A))} \\
      \Phi (\id_{\spe (A)})		&= \G \inv (\id_{\spe (A)} \circ \G_A) \\
				    &= \G \inv (\G_A) = A \\
      \Phi (\overline{\id}_{\spe (A)}) 	&= A^*.
   \end{align*}
  $\Phi$ gives us the possibility to identify functions on the closure of polynomials in $z, \overline{z}$ on $\spe (A)$ with elements in $A$. By Stone-Weierstrass functions on the closure of polynomials in $z, \overline{z}$ on $\spe (A)$ are just continous functions on $\spe (A)$. $\Phi$ is completely determined by its values on $1_{\spe A}$ and $id_{\spe A}$
\end{rem}

\begin{expl}
 Let $\B = \lh$ be the space of bounded linear opeartors on some Hilbert space $\hs$, $T \in \B$ a normal element and $\A$ the star-subalgebra generated by $T$. $\spe (T) \subset \C$ is a compact subset. Any entire function $f : \C \to \C$ is continous on $\spe (A)$, and hence gives us an element $f(A) \in \A$. 
 
 In general, the square root does not give a holomorphic function on $\spe (A)$. However for normal $A$, we can still define a continous square root on $\spe (A)$:
 \[
  \sqrt{\phantom{z}} : \spe (A) \to \C, ~ z \mapsto 
  \begin{cases}
    \sqrt{z} 	& \text{if $z > 0$,} \\
    i \sqrt{-z}	& \text{if $ z < 0$,} \\
    0		& \text{if $ z = 0$.}
  \end{cases}
 \]

 Another interesting example is the absolute value. For real numbers well known, one now has the possibility to take the absolute value of an operator.
\end{expl}

\begin{prop}
 Let $\B$ be an involutive, unitary Banach algebra, $\A$ a unitary star-algebra, and let
\[
  \Phi : \B \to \A 
\]
be an involutive algebra homomorphism.
Then $\Phi$ is continous and norm decreasing.
\end{prop}
\begin{proof}
 Let $B \in \B$. We have
\[
  \spe_\A (\Phi (B)) \subset \spe_\B (B),
\]
since $\Phi(I)=I$. For the spectralradius one has
\[
 \rho (\Phi (B)) \leq \rho ( B) \leq \| B \|.
\]
And consequently
\begin{align*}
 \| \Phi (B) \|^2 &= \| (\Phi (B))\str \Phi (B) \| \\
		  &= \| \Phi (B \str B) \| \\
		  &= \rho ( \Phi(B \str B)) \leq \| B\str B \| \leq \| B \| ^2.
\end{align*}

This gives 
\[
 \| \Phi \| \leq 1.
\]
\end{proof}
\begin{cor}
 Using the same notation as before, the isomoprhism
 \[
  \Phi : \cf ( \spe \A) \to \A \text{ , } f \mapsto \G \inv (f \circ \G_A)
 \]
is the only involutive algebra homomorphism, with the property that \vspace{10 pt} $ \Phi ( 1_{\spe A} ) =  I$  and $\Phi ( \id _{ \spe A }) = A$.
 
\end{cor}

\begin{proof}
 If $\Psi : \cf ( \spe A) \to \A$ is another algebra homomorphism with the properties above, then $\Psi = \Phi$ on all polynomials in $z$ and $\overline{z}$ on $\spe (A)$. We already know that both homomorphisms are continous, hence by Stone-Weierstrass they must coincide on $\cf ( \spe A )$
\end{proof}





