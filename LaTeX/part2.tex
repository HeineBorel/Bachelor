
\section{Spectral Theorem for Bounded Opearators}
% We want to construct a most general spectral theorem for linear opeartors
% between Hilbert spaces. To achieve said goal, we need some auxilliary
% results, derived by Gelfand's theorem. For each normal operator 
% $T \in \lh$, we will consider the associated normal involutive subalgebra 
% \[
%  \A(T) \coloneqq \gen{I, A, B, B^*},~ A \coloneqq \frac{1}{1 + T \str T}, ~ B \coloneqq TA
% \]
% which is well defined, since $1 + T \str T$ is invertible. Gelfand gives us a somewhat o.k. understanding of bounded linear operators, so we might hope to get information about $T$ if we use our understanding of $A$ and $B$, and reconstruct $T$ via $T = BA\inv$.
% First some corollaries from the Gelfand representation theorem
In this chapter, we use the Gelfand transform to prove a continuous
spectral theorem for bounded normal operators, and some auxiliary results
about the spectrum of $C^*$-algebras.
\begin{prop} \label{specinvariant}
 Let $\A$ be a unital $C^*$-subalgebra of the $C^*$-algebra
 $\B$. 
 Let $A$ be an element of $\A$, which is invertible in $\B$.
 Then $A$ is already invertible in $\A$. In other words 
\[
   \spe_{\A}(A) = \spe_{\B}(A) \text{ for all } A \text{ in } \A.
\]

\end{prop}
\begin{proof}
 First assume $A = A^* $. We have $\spe_\A(A) \subset \R$, implying that
  $(A + i \lambda I)$  is invertible in  $\A$, for all $\lambda \neq 0$.
  As 
 \[
  \lim_{\lambda \to 0} (A + i \lambda I) = A,
 \]
 by continuity of the inverse map, and the assumption that $A$ is invertible,
 we get
 \[
  \lim_{\lambda \to 0} (A + i \lambda I) \inv =A \inv.
 \]
Because $ (A + i \lambda I)\inv$ is an element of $\A$ for all $\lambda \neq 0$, 
the statement  holds for self-adjoint $A$, as $\A$ is a closed subalgebra. 

For more general $A$, we consider the self-adjoint element
$A\str \! A$, with inverse  $(A\str \! A)\inv = A\inv (A\inv)^*$. 
Since $\A$ is an involutive algebra, $A\str \! A$ is an element of $\A$, which implies that
$A$ is left-invertible in $\A$, with inverse $(A\str \! A)\inv A^*$.
Using the same argument with the normal element $AA^*$, one gets the
right-invertibility of $A$. Thus $A$ is invertible, and both
inverses coincide.
\end{proof}

\begin{cor}
 Let $\A$ be a unital $C^*$-subalgebra of $\bh$, $T \in \A$. Then 
\[
 \spe_\A (T) = \spe_{\bh}(T) = \spe(T).
\]
\end{cor}

\begin{prop}[Functional calculus for normal elements]\label{boundedfunccalc}
 Let $\B$ be a $C^*$-algebra with unit, and $A$ a normal element.
 Then the algebra $\A = \gen{A, I}$ generated by $A$ and the identity $I$, is a
 commutative involutive subalgebra, which is isomorphic to 
 $\cf (\spe A)$, where $\spe(A)$ is identified with $\spe(\A)$ via the Gelfandtransform.
\end{prop}

\begin{proof}
 First, we show that $\G_A \colon \spe(\A) \to \spe(A) \subset \C$ is injective. Let 
 $\chi_1 , \chi_2$ be elements of $\spe(\A)$. If $\G_A(\chi_1) = \G_A(\chi_2)=
 \chi_2(A)=\chi_1(A)$, then $\chi_1(A^*)=\chi_2(A^*)$ also. Since 
 $\chi_1(I)=\chi_2(I)=1$, we see that $\chi_1 = \chi_2$ on all polynomials
 in $A$ and $A^*$. Because $\chi_1$, $\chi_2$ are continous, they have to coincide on
 $\A$. 
 
 By Proposition \ref{SpecSurj}, $\G_A$ is surjective. Hence, $\G_A$ is a continous bijection 
 from $\spe(\A)$ to $\spe(A)$. As $\spe(\A)$ is compact and $\spe(A)$ Hausdorff,
 $\G_A$ is a homeomorphism.
By the theorem of Gelfand--Neimark 
\[
 \G \colon \A \to \cf(\spe \A)
\]
is an isomorphism. We get the following commutative diagram, which yields the result:

% \textbf{Mit xymatrix}
% \[
% \xymatrix{
%  \A \ar[rrr]^{{B \mapsto \G_B}} \ar[drrr]_{B \mapsto \G_B \circ \G_A \inv~~~} &&& \cf(\spe(\A)) \ar[d]^{\G_B \mapsto \G_B \circ \G_A \inv} \\
%    &&&  \cf(\spe(A))}
% \]
% 
% 
%  
% \textbf{Mit Tikzpicture}
% 
% \begin{center}
%  \begin{tikzpicture}[node distance = 2cm]
%   \node (frA) {$\A$};
%   \node (CSpfrA) [right= 3cm of frA] {$\cf (\spe \A)$};
%   %\node (pol) [below left= 2em and 2em of PWn] {$\Pol_k(W,V)$};
%   \node (CSpA) [below=1.3cm of CSpfrA] {$\cf (\spe A)$};
%   
%   \draw[->] (frA) to node [above] {$B \mapsto \G_B$} (CSpfrA);
%   \draw[->] (frA) to node [very near end, sloped,  below left]{$B \mapsto \G_B \circ \G_A \inv$} (CSpA);
%   \draw[->] (CSpfrA) to node  [right]{$\G_B \mapsto \G_B \circ \G_A \inv$} (CSpA);
%  \end{tikzpicture}
%  \end{center}
\pagebreak
 \begin{center}
 \begin{tikzpicture}[node distance = 2cm]
  \node (frA) {$\A$};
  \node (CSpfrA) [right=  of frA] {$\cf (\spe \A)$};
  %\node (pol) [below left= 2em and 2em of PWn] {$\Pol_k(W,V)$};
  \node (CSpA) [below=1cm of CSpfrA] {$\cf (\spe A)$};
  
  \draw[->] (frA) to node [above] {$\scriptstyle B \mapsto \G_B$} (CSpfrA);
  \draw[->] (frA) to node [ below left]
  {$\scriptstyle B \mapsto \G_B \circ \G_A \inv$} (CSpA);
  \draw[->] (CSpfrA) to node  [right]
  {$\scriptstyle \G_B \mapsto \G_B \circ \G_A \inv$} (CSpA);
 \end{tikzpicture}
 .
 \end{center}

 




\end{proof}

\begin{rem}
  Let $\Phi \colon \cf (\spe A) \to \A$ be the inverse of the isomorphism from the
  previous theorem defined by $B \mapsto \G_B \circ \G_A \inv$.
   For $f \in \cf(\spe A)$, we get 
  \[
   \Phi (f) = \G \inv (f \circ \G_A).
  \]
  Thus, one retrieves the generators of $\A$ via
   \begin{align*}
      \Phi (1_{\spe(A)})   		&= \G \inv (1 _{\spe(A)} \circ \G_A ) \\
				    &= \G \inv (1_{\spe(\A)}), \\
      \Phi (\id_{\spe (A)})		&= \G \inv (\id_{\spe (A)} \circ \G_A) \\
				    &= \G \inv (\G_A) = A, \\
      \Phi (\overline{\id}_{\spe (A)}) 	&= A^*.
   \end{align*}
  The map $\Phi$ gives us the possibility to identify functions on the closure of
  polynomials in $z, \overline{z}$ on $\spe (A)$ with elements in $A$. 
  By the theorem of Stone-Weierstrass, the closure of polynomials in 
  $z, \overline{z}$ on $\spe (A)$ are all continuous functions on $\spe (A)$.
  Furthermore  $\Phi$ is completely determined by its values on $1_{\spe A}$ 
  and $id_{\spe A}$.
\end{rem}

\begin{expl}\label{ExplFuncCalc}
 Let $\B = \bh$ be the space of bounded linear operators on some Hilbert
 space $\hs$, $T$ a normal element and $\A$ the $C^*$-algebra
 generated by $T$.  Any entire function $f \colon \C \to \C$ is continuous on
 $\spe (A)$, and hence gives us an element $f(A)$ in $\A$. 
 
 In general, a complex square root does not give a holomorphic function on
 $\spe (A)$. However for self-adjoint $A$, we can still define a continuous 
 square root, as $\spe(A)$ consists only of real numbers.
 \[
  \sqrt{\phantom{z}} \colon \spe (A) \to \C, ~ z \mapsto 
  \begin{cases}
    \sqrt{z} 	& \text{if $z > 0$,} \\
    i \sqrt{-z}	& \text{if $ z < 0$,} \\
    0		& \text{if $ z = 0$.}
  \end{cases}
 \]

 If the normal operator $T$ is invertible, 0 is not an element of the 
 spectrum. Since the spectrum is a closed subset of $\C$, there is a 
 neighborhood $U$ containing 0, such that $U$ does not intersect $\spe(T)$.
 Thus $f(x) =\nicefrac{1}{x}$ is a continuous function on $\spe(T)$. 
 Since we have $1 = x f(x)$, the spectral theorem implies that $f$
 corresponds to $T\inv$.
 
\end{expl}

\begin{prop}\label{conthomo}
 Let $\B$ be an involutive, unital Banach algebra, $\A$ a unital $C^*$-algebra, and let
\[
  \Phi \colon \B \to \A 
\]
be an involutive algebra homomorphism.
Then $\Phi$ is continuous and norm decreasing.
\end{prop}
\begin{proof}
 Let $B \in \B$. We have
\[
  \spe_\A (\Phi (B)) \subset \spe_\B (B).
\]
For the spectral radius, one therefore has
\[
 \rho (\Phi (B)) \leq \rho ( B) \leq \| B \|.
\]
Consequently
\begin{align*}
 \| \Phi (B) \|^2 &= \| (\Phi (B))\str \Phi (B) \| \\
		  &= \| \Phi (B \str B) \| \\
		  &= \rho ( \Phi(B \str B)) \leq \| B\str B \| \leq \| B \| ^2.
\end{align*}
This gives $ \| \Phi \| \leq 1$.
 \end{proof}

\begin{cor}
 Using the same notation as before, the isomorphism
 \begin{gather*}
  \Phi \colon \cf ( \spe \A) \to \A \text{ , } f \mapsto \G \inv (f \circ \G_A)
 \intertext{
is the only $C^*$-algebra homomorphism with the property that}
 \Phi ( 1_{\spe A} ) =  I \text{  and }\Phi ( \id _{ \spe A }) = A.
 \end{gather*}
\end{cor}

\begin{proof}
 If $\Psi \colon \cf ( \spe A) \to \A$ is another algebra homomorphism with the 
 properties above, then $\Psi = \Phi$ on all polynomials in $z$ and 
 $\overline{z}$ on $\spe (A)$. By Proposition \ref{conthomo}, we know that both homomorphisms are
 continuous. Thus  they must coincide on $\cf ( \spe A )$, by the theorem
 of Stone-Weierstrass.
\end{proof}





