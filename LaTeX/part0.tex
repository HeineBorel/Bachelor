\noindent{\Large \textbf{Einleitung}}
\vspace{.35cm}

\noindent
\begin{otherlanguage}{ngerman}
Ein Spektraltheorem oder Spektralkalkül gibt der Anschauung, dass man 
Operatoren auf einem Hilbertraum in Funktionen einsetzen kann, eine 
rigorose mathematische Grundlage.
Dass man Operatoren in Polynome einsetzen kann, und immer noch sinnvolle
Ausdrücke entstehen, liegt auf der Hand. Wie verhält es sich jedoch mit
stetigen oder gar messbaren Funktionen? Ergibt der Ausdruck $f(T)$ für
beliebige Funktionen $f$ und Operatoren $T$ überhaupt Sinn? Was ist 
$\dom(f(T))$ und $\ran(f(T))$? Ist $f(T)$ dicht definiert?
Diese Fragen möchte ich in dieser Arbeit, soweit es geht, beantworten.

Das Ziel dieser Arbeit ist ein Spektraltheorem für unbeschränkte normale 
Operatoren. Aus diversen Vorlesungen an der Universität Bonn waren mir
Spektraltheoreme für explizite Klassen von Operatoren bekannt, zum 
Beispiel für kompakte, selbstadjungerte Operatoren. Als ich in einem Seminar
eine Variante für unbeschränkte Operatoren benutzen musste, entschied ich
mich mehr mit diesem Thema zu beschäftigen. Diese Arbeit
ist an Studenten der Mathematik oder Physik gerichtet, welche eine 
mathematisch rigorose Formulierung des Spektraltheorems
für unbeschränkte Operatoren kennen lernen möchten.

In meiner Bachelorarbeit wird in Kapitel 2 mit Hilfe des Gelfandschen 
Transformationsatzes ein Spektraltheorem für beschränkte normale Operatoren 
bewiesen. In Kapitel 3 wird versucht, die Methoden des voran gegangenen Kapitels
auf unbeschränkte Operatoren zu erweitern. Um dies zu tun, muss das 
Spektraltheorem auf messbare Funktionen erweitert werden.
Dazu wird in Kapitel 4 beschrieben wie sich Erweiterungen des 
Spektraltheorems auf messbare Funktionen verhalten, das heißt, welche Klassen
von Operatoren erhalten werden. Abschließend werden die Ergebnisse der 
vorherigen Kapitel benutzt, um das Spektraltheorem für unbeschränkte normale
Operatoren zu beweisen, und in Kapitel 6 mit ein paar Beispielen erläutert.

In der Literatur werden zum Beweis von Spektraltheoremen oft, operatorwertige
Maße auf bestimmten $\sigma$--Algebren genutzt.
Dies wird in dieser Arbeit explizit nicht genutzt. Welche der Möglichkeiten
man benutzt, bleibt der eigenen Vorliebe überlassen. Der Riesz--Markov--Kakutani
Darstellungssatz (\cite[Theorem 6.3.4]{PedAnaN})
zeigt, dass beide Herangehensweisen
lediglich zwei Seiten der gleichen Medaille sind. 

Ich möchte mich an dieser Stelle bei meinem Betreuer Herrn Professor Lesch bedanken.
Erst durch seine Betreuung und stetige Erreichbarkeit wurde diese Arbeit ermöglicht.
Weiterhin danke ich meinen Kommilitonen für das Korrekturlesen dieser Arbeit. 
Zum Schluss möchte ich meinen Eltern für die Unterstützung meines Studiums danken.
Ohne Sie, wäre diese nie zu Stande gekommen.


\end{otherlanguage}