\section*{Einleitung}

\addcontentsline{toc}{section}{Einleitung}

\begin{otherlanguage}{ngerman}
Ein Spektraltheorem oder Spektralkalkül gibt der Anschauung, dass man 
Operatoren auf einem Hilbertraum in Funktionen einsetzen kann, eine 
rigorose mathematische Grundlage.
Dass man Operatoren in Polynome einsetzen kann, und immer noch sinnvolle
Ausdrücke entstehen, liegt auf der Hand. Wie verhält es sich jedoch mit
stetigen oder gar messbaren Funktionen? Ergibt der Ausdruck $f(T)$ für
beliebige Funktionen $f$ und Operatoren $T$ überhaupt Sinn? Was ist 
der Definitionsbereich und Bild von $T$? Ist $f(T)$ dicht definiert?
Diese Fragen möchte ich in dieser Arbeit, soweit es geht, beantworten.

Das Ziel dieser Arbeit ist ein Spektraltheorem für unbeschränkte normale 
Operatoren. Aus diversen Vorlesungen an der Universität Bonn waren mir
Spektraltheoreme für explizite Klassen von Operatoren bekannt, zum 
Beispiel für kompakte, selbstadjungerte Operatoren. Als ich in einem Seminar
eine Variante für unbeschränkte Operatoren benutzen musste, entschied ich
mich mehr mit diesem Thema zu beschäftigen. Diese Arbeit
ist an Studenten der Mathematik oder Physik gerichtet, welche eine 
mathematisch rigorose Formulierung des Spektraltheorems
für unbeschränkte Operatoren kennen lernen möchten.

In meiner Bachelorarbeit wird in Kapitel 2 mit Hilfe des Gelfandschen 
Transformationsatzes ein Spektraltheorem für beschränkte normale Operatoren 
bewiesen. In Kapitel 3 wird versucht, die Methoden des voran gegangenen Kapitels
auf unbeschränkte Operatoren zu erweitern. Um dies zu tun, muss das 
Spektraltheorem auf messbare Funktionen erweitert werden.
Dazu wird in Kapitel 4 beschrieben wie sich Erweiterungen des 
Spektraltheorems auf messbare Funktionen verhalten, das heißt, welche Klassen
von Operatoren erhalten werden. Abschließend werden die Ergebnisse der 
vorherigen Kapitel benutzt, um in Kaptiel 5 das Spektraltheorem für unbeschränkte normale
Operatoren zu beweisen, und in Kapitel 6 einige Anwendungen aufgezeigt.

In der Literatur werden zum Beweis von Spektraltheoremen oft, operatorwertige
Maße auf bestimmten $\sigma$--Algebren genutzt.
Dies wird in dieser Arbeit explizit nicht genutzt. Welche der Möglichkeiten
man benutzt, bleibt der eigenen Vorliebe überlassen. Der Riesz--Markov--Kakutani
Darstellungssatz (\cite[Theorem 6.3.4]{PedAnaN})
zeigt, dass beide Herangehensweisen
lediglich zwei Seiten der gleichen Medaille sind. 

Ich möchte mich an dieser Stelle bei meinem Betreuer Herrn Professor Lesch bedanken.
Erst durch seine Betreuung und stetige Erreichbarkeit wurde diese Arbeit ermöglicht.
Weiterhin danke ich meinen Kommilitonen für das Korrekturlesen dieser Arbeit. 
Zum Schluss möchte ich meinen Eltern für die Unterstützung meines Studiums danken.
Ohne Sie, wäre diese nie zu Stande gekommen.


\end{otherlanguage}

\newpage
\section*{Introduction}
\addcontentsline{toc}{section}{Introduction}
A spectral theorem or spectral calculus is a mathematical justification 
of the idea that one can plug operators into functions. One can easily see
that for polynomials, it makes sense to plug in operators. But what about
continuous or even measurable functions? Does the expression $f(T)$ make 
sense for arbitrary operators? What is the domain and range of the corresponding
operators? Is $f$ densely defined? In this text, I want to answer these questions.

The goal of this theses, is to develop a spectral theorem for unbounded normal
operators. From different lectures at the university of Bonn, I knew a few explicit
formulations of spectral theorems about compact self-adjoint operators. Preparing
for a seminar, I was forced to work with a spectral theorem for unbounded operators, 
and decided that I want to understand this subject properly. This theses is aimed
at students of mathematics and physics, which want to see a rigorous statement
of the spectral theorem for unbounded operators.

In Chapter 2, we will construct a continuous spectral theorem for bounded operators
using the Gelfand representation theorem. Following this, we try to extend this to unbounded
operators in Chapter 3. To do this, we have to extend the spectral calculus to measurable
functions, done in Chapter 4. In Chapter 5, we use the previous results to develop a spectral
calculus for unbounded normal operators. The last chapter will show a few applications.

To prove a spectral theorem, one often encounters projection valued measures. We will not 
use these, in this theses, but rather use operator valued functionals.
However, it is irrelevant, what specific tool you choose:
By the Riesz--Markov--Kakutani representation theorem, both are equivalent choices.

At this point, I want to thank Professor Lesch. His constant availability made this theses possible.
Furthermore, I want to thank my fellow students for proof-reading this work. In the end, I thank
my parents for always supporting me. Without them, this work would not even have been started.



