{\Large \textbf{Einleitung}}
\vspace{.35cm}

Ein Spektraltheorem oder Spektralkalkül gibt der Anschauung, dass man 
\unsure{Vielleich Endomorphismen von Hilbertraumen}
Operatoren in Funktionen einsetzen kann, eine 
rigorose mathematische Grundlage.
Dass man Operatoren in Polynome einsetzen kann, und immer noch sinnvolle
Ausdrücke entstehen, liegt auf der Hand. Wie verhält es sich jedoch mit
stetigen oder gar messbaren Funktionen? Ergibt der Ausdruck $f(T)$ für
beliebige Funktionen $f$ und Operatoren $T$ überhaupt Sinn? Was ist 
$\dom(f(T))$ und $\ran(f(T))$? Ist $f(T)$ dicht definiert?
Diese Fragen möchte ich in dieser Arbeit, soweit es geht, beantworten.

Das Ziel dieser Arbeit ist ein Spektraltheorem für unbeschränkte normale 
Operatoren. Aus diversen Vorlesungen an der Universität Bonn waren mir
Spektraltheoreme für explizite Klassen von Operatoren bekannt, zum 
Beispiel für kompakte, selbstadjungerte Operatoren. Als ich in einem Seminar
eine Variante für unbeschränkte Operatoren benutzen musste, entschied ich
mich mehr mit diesem Thema zu beschäftigen. Diese Arbeit
ist an Studenten der Mathematik oder Physik gerichtet, welche eine 
mathematisch rigorose Formulierung, \unsure{Umformulieren?}und Beweis des Spektraltheorems
für unbeschränkte Operatoren kennen lernen möchten.

In meiner Bachelorarbeit wird in Kapitel 2 mit Hilfe des Gelfandschen 
Transformationsatzes ein Spektraltheorem für beschränkte normale Operatoren 
bewiesen. In Kapitel 3 wird versucht, die Methoden des voran gegangenen Kapitels
auf unbeschränkte Operatoren zu erweitern. Um dies zu tun, muss das 
Spektraltheorem auf messbare Funktionen erweitert werden.
Dazu wird in Kapitel 4 beschrieben wie sich Erweiterungen des 
Spektraltheorems auf messbare Funktionen verhalten, das heißt, welche Klassen
von Operatoren erhalten werden. Abschließend werden die Ergebnisse der 
vorherigen Kapitel benutzt, um das Spektraltheorem für unbeschränkte normale
Operatoren zu beweisen, und in Kapitel 6 mit ein paar Beispielen erläutert.

In der Literatur werden zum Beweis von Spektraltheoremen oft, sogenannte 
Spektralmaße benutzt.
Dies wird in dieser Arbeit explizit nicht genutzt. Welche der Möglichkeiten
man benutzt, bleibt der eigenen Vorliebe überlassen. Beide Herangehensweisen
sind lediglich zwei Seiten der \unsure{Hier noch Literatur raussuchen}
gleichen Medaille.