\section{Unbounded Operators}\label{section3}



Since we have a functional calculus for normal bounded operators, one might hope 
that we can extend our results to unbounded operators. But the previous 
result relied on the Gelfand transform, which in turn relied on the existence
of certain structures, such as the operator being an element in an algebra. 
But as previously remarked, closed operators are not that nice.
This chapter follows \cite{LesHaupt}. Lemma \ref{BBounded} is taken from 
\cite[p. 319]{ConFuncAna}.

From now on, let $T \in \lh$ be a closed normal operator. 
We endow $\domt$ with the graph scalar product, where 
$\iota\colon \dom(T) \hookrightarrow \hs$ is the canonical inclusion.
\[
 \gen{x,y}_T \coloneqq  \gen{\iota x, \iota y}_\hs +
 \gen{ T \iota x, T \iota y}_\hs,
\]
This turns $\dom(T)$ into a Hilbert space. The topology given by the graph scalar product
is finer than the subspace topology, as convergence in the graph norm
 implies convergence in the subspace topology. Furthermore,
$T$ seen as a map from $( \domt , \gen{\cdot, \cdot}_T)$ to $\hs$, is 
continous.

If there is no room for misinterpretation,
we will omit the $\hs$ in the scalar product. The adjoint of $T$ as a closed 
operator from $\hs$ to itself, will be called $T^*$. 
We have two ways to interpret the inclusion map $\iota$:
\begin{enumerate}
 \item as an operator on $\hs$, namely the identity with domain $\domt$, or
 \item as a bounded linear operator $\iota \colon ( \domt , \gen{\cdot, \cdot}_T)
 \to ( \hs , \gen{\cdot, \cdot}_\hs)$.
\end{enumerate}
Using the second interpretation, we note that by 
$\ran(\iota\str) = \ker(\iota)^\perp =\hs$
and $\ker(\iota\str) = \ran(\iota)^\perp = 0$, $\iota\str$ is a bijection.
\begin{lem}\label{TTstrDense}
 $\dom(T\str T)$ is dense in $( \domt , \gen{\cdot, \cdot}_T)$.
\end{lem}
\begin{proof}
 We show that $\dom(T \str T)^{\perp_T} = 0$. Let $x \in \dom(T \str T), 
 ~y \in \dom(T \str T)^{\perp_T}$. Then
 \begin{align*}
  0 &=\gen{x, y}_T = \gen{\iota x, \iota y}_\hs + \gen{T\iota x, T\iota y}
  _\hs \\
  &=\gen{\iota x, \iota y}_\hs + \gen{T\str T\iota x, \iota y}_\hs
= \gen{(I + T \str T) \iota x, \iota y}_\hs.
 \end{align*}
Therefore, if we prove $\ran(I + T\str T)$ is dense in $\hs$, the claim is
proven as well. Since $\ran(I + T\str T)^{\perp_\hs} = \ker(I + T\str T)$, we prove
injectivity of $(I + T\str T)$,
\begin{align*}
 \| (I + T\str T)x \| ^2  
 &= \|x \| ^2 + \| T \str T x\| ^2 + \gen{x, T\str T x}
 + \gen{T \str T x, x}\\
 &=\|x \| ^2 + \| T \str T x\| ^2 + 2\gen{Tx,  T x} \\
 &= \|x \| ^2 + \| T \str T x\| ^2 + 2\|T x\|^2 \geq \|x\|^2.
\end{align*}

\end{proof}


\begin{prop}
 $(I + T \str T)$ is boundedly
 invertible, with inverse $\iota \iota \str$.
\end{prop}
\begin{proof}
  The calculation of the last lemma shows that $(I + T \str T)$ is injective.
  Therefore, it suffices to show that there exists a right inverse.
  Let $x \in \domt$, $y \in \hs$ such that 
  $\iota  \iota \str y \in \dom(T\str T)$.
  We see that
  \begin{align*}
\gen{\iota x, y}_\hs &= \gen{x, \iota \str y}_T = 
  \gen{\iota x,\iota \iota \str y}_\hs +
  \gen{ T \iota x, T \iota \iota \str y}_\hs \\
  &=\gen{\iota x, \iota \iota \str y}_\hs  +
  \gen{\iota x , T \str T \iota \iota \str y}_\hs  =
  \gen{\iota x, \iota \iota \str y + T\str T \iota \iota \str y}_\hs .
  \end{align*}
    Substracting the left hand side, we get
  \[
   0 = \gen{\iota x , \iota \iota \str y + T\str T \iota \iota \str y - y}.
  \]
  Note, that since $\iota \str$ is a continuous bijection and $\dom(T \str T)
  \subset \dom(T)$ is a
  dense subset of both $\dom(T)$ and $\hs$, $(i \str) \inv (\dom(T \str T))$ is dense
  in $\hs$. As this equality holds for all $x \in \domt$, we get
  \begin{align*}
  y=\iota \iota \str y + T\str T \iota \iota \str y 
  = (1+ T \str T) \iota \iota \str y,
  \end{align*}
  which implies that $ \iota \iota \str = ( I + T \str T) \inv$,
  as $\iota \iota\str $ is bounded and $(I = T\str T)$ is injective.
\end{proof}

Define $A \coloneqq \iota \iota \str$ and $B \coloneqq TA$. 
If we think of $A$ corresponding to $\nicefrac{1}{1 + |x|^2}$, then we would expect
$B$ to be bounded as well.
As it turns out, this is true.
\begin{lem}\label{BBounded}
 $B = TA = T(I + T \str T)\inv$ is a bounded operator, and we have
 $AT \subset TA$.
\end{lem}



\begin{proof}
 Let $x \in \dom(I + T \str T)$ such that $(I + T \str T)x = y \in \domt$.
 Using the calculation at the end of Lemma \ref{TTstrDense}, we get 
 $\| y + T\str T y \|^2 \geq \|Ty\|^2$.
From that, $\| TAy\|^2 = \| Tx \| ^2 \leq \| (I + T \str T)x \|^2 = \| y \| ^2$
, which proves that $B$ is bounded. 

To show that $AT \subset TA$, take $y \in \dom(AT) = \domt$,
$x \in \dom(T \str T)$ such that $y = ( I +  T \str T)x$. $T \str T x \in \domt$
which implies $Tx \in \dom(TT\str) = \dom(T\str T)$. Then 
\[
 ATy = A(Tx + T T\str T x) = A(I + T\str T)Tx= Tx,
\]
and
\[
 TAy= T(I+ T\str T)\inv (I + T \str T)x=Tx,
\]
concluding that $AT = TA$ on $\domt$.

\end{proof}

The operator $AT$ is bounded but not defined on all of $\hs$. So we extend it
in the following

\begin{lem}
 $AT$ admits a bounded linear extension $\overline{AT}$ to all of $\hs$. 
 We then have $\overline{AT}=TA$.
\end{lem}

\begin{proof}
 For $x \in \hs$ we can choose a sequence $x_n \in \domt$, with 
 $x_n \rightarrow x$. Define $\overline{AT}(x) \coloneqq TA(x)$.
 This is linear, because $AT = TA$ on $\domt$. As $TA$ is linear bounded,
 the limit does not depend on the chosen sequence.
\end{proof}

\begin{rem}
 The previous two lemmata and their proofs, still hold if we replace $T$ by $T\str$,
 giving us 
 \[
  AT\str = T\str \! A \text{ and hence } B\str = T\str \! A.
 \]
 One also has the identity
 \begin{align*}
  A^2 + B\str B &= (I + T\str T)^{-2} + T \str (I + T\str T)
		    \inv T (I + T\str T) \inv \\
		&= (I + T\str T)^{-2} + T \str T(I + T\str T)
		   \inv (I + T\str T) \inv \\
		&= (I + T\str T)(I + T\str T)^{-2} \\
		&= (I + T\str T)\inv  \\
		&= A .
  \end{align*}

  

\end{rem}

From now on, we identify $B={AT}$ and $B\str ={AT\str}$
with their bounded extensions.
Define $\A = \A(T)$ by $\A \coloneqq  \gen{I, A, B }$. Let $ \chi \in \spe (\A)$
, such that $\chi(A) =0$. By the previous identity, we get
\[
 \chi(A)^2 + |\chi(B)|^2 = \chi(A),
\]
which implies that $\chi(B) =0$ as well. But for all $\chi$ in  $\spe (\A)$, 
 it holds that $\chi(I)=1$.
If such a $\chi$ exists, it is therefore unique. We call this
character $\chi_\infty$.

Define $\theta \colon \spe (\A) \to \cb$ by

 \[
 \chi \mapsto 
  \begin{cases}
    \frac{\chi(B)}{\chi(A)} &, \text{ if }\chi \neq \chi_\infty\\
    \infty &, \text{ if } \chi = \chi_\infty.
    \end{cases}
 \]
 Let $\chi \neq \chi_\infty$. Since $\chi$ is a involutive algebrahomomorphism,
 $A^2 + B\str B = A$ implies for
   $\chi(A) = \chi(A) \overline{\chi(A)} + \chi(B) \overline{\chi(B)} $
, which is equivalent to  
  $ \frac{1}{\chi(A)} = 1 + \frac{\chi(B)}{\chi(A)} 
  \overline{\left( \frac{\chi(B)}{\chi(A)} \right)} = 1 + | \theta(\chi)|^2$.
 The last equalities hold because $A$ is self-adjoint, implying that $\chi(A)$ is a real number.
 Inverting the last equality gives
\[
 \chi(A) = \frac{1}{1 + |\theta (\chi)|^2} \tag{$\ast$}.
\]
The definition of $\theta$ (and not  $T = 
`` \frac{B}{A} ")$, gives
\[
 \chi(B) = \chi(A) \frac{\chi(B)}{\chi(A)}= \chi(A)~ \theta(\chi) 
 \tag{$\ast \ast$}.
\]
Recalling the definition of the Gelfand transform, we see that our map
\[
 \theta \colon \spe(\A) \setminus \setl\chi_\infty\setr \to \C
\]
equals a fraction of two single Gelfand transform
\[
 \theta(\chi) = \frac{\chi(B)}{\chi(A)} = \frac{\G_B (\chi)}{\G_A (\chi)}=
 \frac{\G_B}{\G_A}(\chi).
\]
But $\G_A \neq 0$ on $\spe(\A) \setminus \setl\chi_\infty\setr$, which implies
that $\theta$ is continuous on $\spe(\A) \setminus \setl\chi_\infty\setr$. 
To show that $\theta$ is continuous at $\chi_\infty$
, let $ \left( \chi_\lambda \right) 
_{\lambda \in \Lambda}$ be a net converging to $\chi_\infty$ and 
$\chi_\lambda \neq \chi_\infty$ for all $\lambda \in \Lambda$. If 
no such net exists, we need not worry about continuity as $\chi_\infty$
does not even exist, or is an isolated point in $\spe(\A)$.
By continuity of $\G_A$ we have
\[
 \G_A (\chi_\lambda) = \chi_\lambda (A) \to \chi_\infty (A) = 0.
\]
Equation $(\ast)$ implies
\[
 | \theta( \chi_\lambda ) | ^2  + 1 = \frac{1}{\chi(A)} \to \infty,
\]
which is equivalent to 
\[
 | \theta (\chi_\lambda ) | \to \infty.
\]
This implies that $\theta$ is continuous. 
Summarizing the previous pragraph, we get 
\begin{lem}
$\theta$ extends to a continuous map $ \theta \colon \spe(\A) \to \cb,$
by $\theta (\chi_\infty) \coloneqq \infty$.
Furthermore $\theta \colon \spe (\A) \to \cb$ is a homeomorphism onto its image.
\end{lem}

\begin{proof}
The first claim was proven before. For the second claim, we check 
that $\theta$ is injective:
Let $\chi_1 , \chi_2 \neq \chi_\infty$. Equations $(\ast)$ and $(\ast \ast)$
imply that, if $\theta (\chi_1)= \theta( \chi_2)$, \newline $\chi_1$ coincides with
$\chi_2$. Furthermore, $\chi_\infty$ is unique, which implies that $\theta$
is injective. Since $\spe (\A)$ is compact and $\cb$ is Hausdorff, 
this proves the second claim.
\end{proof}

Combining the Gelfandisomorphism $\G \colon \A \to \cf (\spe \A)$, with
$\theta$, one has

\begin{align*}
 \A \longrightarrow  \cf (&\spe \A) \longrightarrow  \cf (\theta ( \spe \A )) \\
 x \mapsto  \G_x &~;~ f \mapsto  f \circ \theta \inv \\
 \G \inv f \mapsfrom f &~;~ g \circ \theta \mapsfrom g.
\end{align*}
Define
\[
 \Phi \colon \cf ( \theta (\spe \A)) \to \A,~ g \mapsto \G \inv (g \circ \theta).
\]
As $\G$  is an involutive algebrahomomorphism, $\Phi$ is as well.

\begin{prop} \label{spectraluniqueness}
 Let $T$ be a normal operator on $\hs$, $\A = \gen{I, A, B}$ the $C^*$-algebra
 associated to $T$. Then $\Phi$ is the only involutive algebrahomomorphism from
 $\cf (\theta (\spe \A ))$ onto $\A$, such that
 \[
  \Phi \left(\frac{1}{1 + | z| ^2}\right) = A , ~ 
  \Phi\left(\frac{z}{1+ |z|^2}\right) = B.
 \]
\end{prop}



\begin{proof}
 Let $z \in \theta(\spe\A)$, $z = \theta(\chi)$. Using 
 $\G_A(\chi)=\chi(A)=\frac{1}{1+ |\theta(\chi)|^2}$, we get
 \begin{gather*}
  \Phi \inv (A)(z) = \G_A \circ \theta \inv (z)
		   = \G_A (\theta \inv (z))
		   = \frac{1}{1+ |z|^2}.
\intertext{ Moreover, using $\G_B(\chi) =
\chi(A) \theta(\chi)= \frac{\theta(\chi)}{1 + | \theta(\chi)|^2}$, we get}
\Phi \inv (B)(z) = \G_B \circ \theta \inv (z) 
		   = \G_B (\theta \inv (z)) 
		  = \frac{z}{1 + |z|^2}.
 \end{gather*}
 
 To proof uniqueness, let $\Psi \colon \cf (\theta (\spe \A)) \to \A$ be another
 involutive algebra homomorphism, such that
 \[
  \Psi \left(\frac{1}{1 + | z| ^2}\right) = A ,\text{ and}
  ~ \Psi\left(\frac{z}{1+ |z|^2}\right) = B.
 \]
 Then $\Phi$ coincides with $\Psi$ on all polynomials in $A, B, \overline{B}$.
 Theses polynomials form an involutive
 algebra, which seperates points. By the theorem of Stone-Weierstrass,
 it is dense, and 
 therefore $\Phi$ equals $\Psi$ as they are continious.
\end{proof}

Our goal is to construct a functional calculus for $T$. With respect to $\Phi$,
$T$ would corresponds to 
$\id_{\theta (\spe \A)}$. But if $\chi_\infty \in \spe (\A)$, then 
$\id_{\theta ( \spe \A ))}
\notin \cf (\spe \A)$ since $\infty \notin \C$.




